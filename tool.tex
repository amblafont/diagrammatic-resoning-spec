\documentclass{article}
\usepackage{yade}
\usepackage{hyperref}
\usepackage{amsmath}
\usepackage{ebutf8}

\usepackage[nameinlink]{cleveref}

\newtheorem{example}{Example}[section]
\newtheorem{question}{Question}[section]
\newtheorem{remark}{Remark}[section]
\title{Specification of a diagrammatic reasoning tool}
\author{Ambroise LAFONT}
\begin{document}
\maketitle{}
This is a description of a tool that allows to reason graphically. 
The user interaction process consists of two stages:
\begin{enumerate}
    \item an algebraic specification, that is, specification of an algebraic structure;
    %  (e.g., the structure of categories).
    \item a construction mode 
    to build elements in any model, based on some input data.
%     \footnote{Because of second-order rewriting rules, 
% the notion of "initial model" is unclear.}.
    \end{enumerate}
    \begin{example}[Proof that monomorphisms compose]
        \label{ex:proof-mono}
        The algebraic structure is the structure of categories.
        % equipped with two monomorphisms.
        The graphical reasoning mode would allow to construct the composite of two monomorphisms, and mark it as a monomorphism.
    \end{example}
    \tableofcontents
\section{Algebraic specification}
The algebraic specification consists in two parts:
\begin{itemize}
    \item A sort specification, that specifies what are the different sorts of elements available in the algebraic structure.
    \item A rewriting system specification, that is, a list of rewriting rules.
\end{itemize}

\begin{example}
    \label{ex:proof-mono-alg-spec}
    Continuing \Cref{ex:proof-mono}, the sort specification would be a list of fours sorts: objects, morphisms, commutative triangles, and monomorphisms.
One rewriting rule would consist in completing 
any composable pair of morphism can be completed to a commutative triangle.
\end{example}

From the user perspective, the algebraic specification mainly consists in drawing diagrams, where certain elements are visually "distinguished" (in this document, we will use the red color, but it could be something else).


\subsection{Sort specification}
The sorts are specified one after the other.
Each sort is specified with a diagram, where the undistinguished elements represent the \emph{dependencies} of the sort, 
while the distinguished elements determine how this new sort should be represented.
% When two sorts have the same dependencies, one can be "marked" as a subsort of the other.
\begin{example}
    Let us detail the sort specification of \Cref{ex:proof-mono-alg-spec}.
    \begin{description}
        \item[Objects]
        The sort has no dependencies, and we want to represent them as vertices, so the diagram would be:
         \[
        % YADE DIAGRAM sort-obj.yade
        % GENERATED LATEX
        \begin{tikzpicture}[every node/.style={outer sep=0pt,anchor=base,text height=1.2ex, text depth=0.25ex}] 
\node[inner sep=5pt] (0) at (11.904761904761905em, -7.142857142857143em) {$\textcolor{red}{\bullet}$} ; 
\path 
; 
\path[->] 
; 
\end{tikzpicture}
        % END OF GENERATED LATEX
    \]
    \item[Morphisms]
    The sort has two dependencies: the source and target object. We want to represent a morphism as an arrow, so the diagram is:
    \[
        % YADE DIAGRAM sort-mor.yade
        % GENERATED LATEX
        \begin{tikzpicture}[every node/.style={outer sep=0pt,anchor=base,text height=1.2ex, text depth=0.25ex}] 
\node[inner sep=5pt] (0) at (16.666666666666668em, -7.142857142857143em) {$\bullet$} ; 
\node[inner sep=5pt] (1) at (21.428571428571427em, -7.142857142857143em) {$\bullet$} ; 
\path 
(0) to[fore, red,->, ] node[coordinate](2){} (1) 
; 
\path[->] 
(0) edge["${\scriptstyle }$", pos=0.5, fore, red,->, ] (1) 
; 
\end{tikzpicture}
        % END OF GENERATED LATEX
    \]
    \item[Monomorphisms]
     The sort depends on a morphism, and 
     we want to represent it as an additional "monic" mark on the arrow.
    \[
        % YADE DIAGRAM sort-mono.yade
        % GENERATED LATEX
        \begin{tikzpicture}[every node/.style={outer sep=0pt,anchor=base,text height=1.2ex, text depth=0.25ex}] 
\node[inner sep=5pt] (0) at (11.904761904761905em, -7.142857142857143em) {$\bullet$} ; 
\node[inner sep=5pt] (1) at (16.666666666666668em, -7.142857142857143em) {$\bullet$} ; 
\path 
(0) to[fore, black,->, ] node[coordinate](2){} (1) 
(0) to[fore, red,->, into, ] node[coordinate](3){} (1) 
; 
\path[->] 
(0) edge["${\scriptstyle }$", pos=0.5, fore, red,->, into, ] (1) 
(0) edge["${\scriptstyle }$", pos=0.5, fore, black,->, ] (1) 
; 
\end{tikzpicture}
        % END OF GENERATED LATEX
    \]
    % We also mark this sort as a subsort of the sort of morphisms.
    \item[Commutative triangles] 
     The sort has dependencies triangles. We can represent a commutative diagram as a 2-cell.
    \[
        % YADE DIAGRAM sort-triangle.yade
        % GENERATED LATEX
        \begin{tikzpicture}[every node/.style={outer sep=0pt,anchor=base,text height=1.2ex, text depth=0.25ex}] 
\node[inner sep=5pt] (0) at (11.904761904761905em, -7.142857142857143em) {$\bullet$} ; 
\node[inner sep=5pt] (1) at (16.666666666666668em, -2.380952380952381em) {$\bullet$} ; 
\node[inner sep=5pt] (2) at (21.428571428571427em, -7.142857142857143em) {$\bullet$} ; 
\path 
(0) to[fore, black,->, ] node[coordinate](3){} (1) 
(1) to[fore, black,->, ] node[coordinate](4){} (2) 
(0) to[fore, black,->, ] node[coordinate](5){} (2) 
(1) to[fore, red,->, cell=0.2, ] node[coordinate](6){} (5) 
; 
\path[->] 
(0) edge["${\scriptstyle }$", pos=0.5, fore, black,->, ] (1) 
(1) edge["${\scriptstyle }$", pos=0.5, fore, black,->, ] (2) 
(0) edge["${\scriptstyle }$", pos=0.5, fore, black,->, ] (2) 
(1) edge["${\scriptstyle }$", pos=0.5, fore, red,->, cell=0.2, ] (5) 
; 
\end{tikzpicture}
        % END OF GENERATED LATEX
    \]
    \end{description}

    
\end{example}
\subsection{Rewriting rules}
\subsubsection{First-order rules}
A rewriting rule is specified as a \emph{rewriting diagram}, where the non distinguished part is the pattern. The idea is that when this pattern is found in the workspace (we call this a \emph{match}), the workspace can be extended with the distinguished part.

\begin{remark}
    Intuitively, the rewriting diagram is a rule that says: "if you see these (non distinguished) elements, you can add those (distinguished) elements". Logically, it is a formula $∀ .. ∃ .. $.    
\end{remark}

A rewriting diagram may also include equality tags between elements with the same dependencies (e.g., between two morphisms with the same source and target): those elements will be merged in the extended workspace.

\begin{remark}
    \label{rem:formula-fo}
   Logically, it is a formula $∀ .. ∃ .. P $ where $P$ is a cunjunction of equalities between variables introduced by $∀ $.    
\end{remark}
\begin{remark}
    The unique existence property $∀ \vec{x}∃ !\vec{y}  .. P(\vec{x})$ is equivalent to the conjunction of 
    $ ∀ \vec{x}∃ \vec{y}  P(\vec{x})$ and $∀ \vec{x}/P ∀ \vec{y} ∀ \vec{y}' \vec{y}=\vec{y'} $, where $\vec{x}/P$ is a smaller vector of variables (taking into account the equations in $P$), and thus can be represented by two rewriting diagrams.
    The software should be able to automatically turn 
    a unique existence rewriting diagram into two rewriting diagrams.
\end{remark}
\begin{example}
    Let us detail the first-order rewriting rules of \Cref{ex:proof-mono-alg-spec}.

    \begin{description}
        \item[Composition] 
   
\begin{equation}
% YADE DIAGRAM rew-compo.yade
% GENERATED LATEX
\begin{tikzpicture}[every node/.style={outer sep=0pt,anchor=base,text height=1.2ex, text depth=0.25ex}] 
\node[inner sep=5pt] (0) at (11.857142857142858em, -7.395089285714286em) {$\bullet$} ; 
\node[inner sep=5pt] (1) at (16.61904761904762em, -2.6331845238095237em) {$\bullet$} ; 
\node[inner sep=5pt] (2) at (21.38095238095238em, -7.395089285714286em) {$\bullet$} ; 
\path 
(0) to[fore, black,->, ] node[coordinate](3){} (1) 
(1) to[fore, black,->, ] node[coordinate](4){} (2) 
(0) to[fore, red,->, ] node[coordinate](5){} (2) 
(1) to[fore, red,->, cell=0.2, ] node[coordinate](6){} (5) 
; 
\path[->] 
(0) edge["${\scriptstyle }$", pos=0.5, fore, black,->, ] (1) 
(1) edge["${\scriptstyle }$", pos=0.5, fore, black,->, ] (2) 
(0) edge["${\scriptstyle }$", pos=0.5, fore, red,->, ] (2) 
(1) edge["${\scriptstyle }$", pos=0.5, fore, red,->, cell=0.2, ] (5) 
; 
\end{tikzpicture}
% END OF GENERATED LATEX
\label{eq:rew-comp}
\tag{Comp}
\end{equation}

\item[Monomorphic property]
\begin{equation}
% YADE DIAGRAM rew-mono-fo.yade
% GENERATED LATEX
\begin{tikzpicture}[every node/.style={outer sep=0pt,anchor=base,text height=1.2ex, text depth=0.25ex}] 
\node[inner sep=5pt] (0) at (11.904761904761905em, -7.142857142857143em) {$\bullet$} ; 
\node[inner sep=5pt] (1) at (16.666666666666668em, -7.142857142857143em) {$\bullet$} ; 
\node[inner sep=5pt] (2) at (7.142857142857143em, -7.142857142857143em) {$\bullet$} ; 
\node[inner sep=5pt] (3) at (9.428571428571429em, -7.0617559523809526em) {$\textcolor{red}{=}$} ; 
\path 
(0) to[fore, black,->, into, ] node[coordinate](4){} (1) 
(2) to[fore, black,->, curve={ratio=-0.30000000000000004}, ] node[coordinate](5){} (0) 
(2) to[fore, black,->, curve={ratio=-0.7999999999999999}, ] node[coordinate](6){} (1) 
(2) to[fore, black,->, curve={ratio=0.30000000000000004}, ] node[coordinate](7){} (0) 
(0) to[fore, black,->, cell=0.2, curve={ratio=0.30000000000000004}, ] node[coordinate](8){} (6) 
(0) to[fore, black,->, cell=0.2, curve={ratio=-0.2}, ] node[coordinate](9){} (6) 
; 
\path[->] 
(0) edge["${\scriptstyle }$", pos=0.5, fore, black,->, into, ] (1) 
(2) edge["${\scriptstyle }$", pos=0.5, fore, black,->, curve={ratio=-0.30000000000000004}, ] (0) 
(2) edge["${\scriptstyle }$", pos=0.5, fore, black,->, curve={ratio=-0.7999999999999999}, ] (1) 
(2) edge["${\scriptstyle }$"', pos=0.5, fore, black,->, curve={ratio=0.30000000000000004}, ] (0) 
(0) edge["${\scriptstyle }$", pos=0.5, fore, black,->, cell=0.2, curve={ratio=0.30000000000000004}, ] (6) 
(0) edge["${\scriptstyle }$", pos=0.5, fore, black,->, cell=0.2, curve={ratio=-0.2}, ] (6) 
; 
\end{tikzpicture}
% END OF GENERATED LATEX
\tag{Mono}
\label{eq:rew-mono-fo}
\end{equation}
Note that there is an ambiguity here about the dependencies of the 2-cells. The interface should give a way to make them visually explicit, e.g., using a color convention as follows:
\[
% YADE DIAGRAM disambiguity.yade
% GENERATED LATEX
\begin{tikzpicture}[every node/.style={outer sep=0pt,anchor=base,text height=1.2ex, text depth=0.25ex}] 
\node[inner sep=5pt] (0) at (16.666666666666668em, -11.904761904761905em) {$\bullet$} ; 
\node[inner sep=5pt] (1) at (21.428571428571427em, -11.904761904761905em) {$\bullet$} ; 
\node[inner sep=5pt] (2) at (11.904761904761905em, -11.904761904761905em) {$\bullet$} ; 
\node[inner sep=5pt] (3) at (30.952380952380953em, -11.904761904761905em) {$\bullet$} ; 
\node[inner sep=5pt] (4) at (35.714285714285715em, -11.904761904761905em) {$\bullet$} ; 
\node[inner sep=5pt] (5) at (26.19047619047619em, -11.904761904761905em) {$\bullet$} ; 
\path 
(0) to[fore, red,->, into, ] node[coordinate](6){} (1) 
(2) to[fore, red,->, curve={ratio=-0.30000000000000004}, ] node[coordinate](7){} (0) 
(2) to[fore, red,->, curve={ratio=-0.7999999999999999}, ] node[coordinate](8){} (1) 
(2) to[fore, black,->, curve={ratio=0.30000000000000004}, ] node[coordinate](9){} (0) 
(3) to[fore, red,->, into, ] node[coordinate](10){} (4) 
(5) to[fore, black,->, curve={ratio=-0.30000000000000004}, ] node[coordinate](11){} (3) 
(5) to[fore, red,->, curve={ratio=-0.7999999999999999}, ] node[coordinate](12){} (4) 
(5) to[fore, red,->, curve={ratio=0.30000000000000004}, ] node[coordinate](13){} (3) 
(0) to[fore, black,->, cell=0.2, curve={ratio=0.30000000000000004}, ] node[coordinate](14){} (8) 
(0) to[fore, blue,->, cell=0.2, curve={ratio=-0.2}, ] node[coordinate](15){} (8) 
(3) to[fore, blue,->, cell=0.2, curve={ratio=0.30000000000000004}, ] node[coordinate](16){} (12) 
(3) to[fore, black,->, cell=0.2, curve={ratio=-0.2}, ] node[coordinate](17){} (12) 
; 
\path[->] 
(0) edge["${\scriptstyle }$", pos=0.5, fore, red,->, into, ] (1) 
(2) edge["${\scriptstyle }$", pos=0.5, fore, red,->, curve={ratio=-0.30000000000000004}, ] (0) 
(2) edge["${\scriptstyle }$", pos=0.5, fore, red,->, curve={ratio=-0.7999999999999999}, ] (1) 
(2) edge["${\scriptstyle }$"', pos=0.5, fore, black,->, curve={ratio=0.30000000000000004}, ] (0) 
(3) edge["${\scriptstyle }$", pos=0.5, fore, red,->, into, ] (4) 
(5) edge["${\scriptstyle }$", pos=0.5, fore, black,->, curve={ratio=-0.30000000000000004}, ] (3) 
(5) edge["${\scriptstyle }$", pos=0.5, fore, red,->, curve={ratio=-0.7999999999999999}, ] (4) 
(5) edge["${\scriptstyle }$"', pos=0.5, fore, red,->, curve={ratio=0.30000000000000004}, ] (3) 
(0) edge["${\scriptstyle }$", pos=0.5, fore, black,->, cell=0.2, curve={ratio=0.30000000000000004}, ] (8) 
(0) edge["${\scriptstyle }$", pos=0.5, fore, blue,->, cell=0.2, curve={ratio=-0.2}, ] (8) 
(3) edge["${\scriptstyle }$", pos=0.5, fore, blue,->, cell=0.2, curve={ratio=0.30000000000000004}, ] (12) 
(3) edge["${\scriptstyle }$", pos=0.5, fore, black,->, cell=0.2, curve={ratio=-0.2}, ] (12) 
; 
\end{tikzpicture}
% END OF GENERATED LATEX
\]
Note that we could have added for convenience\footnote{This is to make the output diagram smaller.} an equality tag between 2-cells, 
but it would be more principled 
to have a seperate rewrite rule  to make commutative triangles irrelevant:
\[
% YADE DIAGRAM commutative-proof-irrelevance.yade
% GENERATED LATEX
\begin{tikzpicture}[every node/.style={outer sep=0pt,anchor=base,text height=1.2ex, text depth=0.25ex}] 
\node[inner sep=5pt] (0) at (16.666666666666668em, -11.904761904761905em) {$\bullet$} ; 
\node[inner sep=5pt] (1) at (21.428571428571427em, -7.142857142857143em) {$\bullet$} ; 
\node[inner sep=5pt] (2) at (26.19047619047619em, -11.904761904761905em) {$\bullet$} ; 
\node[inner sep=5pt] (3) at (21.476190476190474em, -9.5625em) {$\textcolor{red}{=}$} ; 
\path 
(0) to[fore, black,->, ] node[coordinate](4){} (1) 
(1) to[fore, black,->, ] node[coordinate](5){} (2) 
(0) to[fore, black,->, ] node[coordinate](6){} (2) 
(1) to[fore, black,->, cell=0.2, curve={ratio=-0.30000000000000004}, ] node[coordinate](7){} (6) 
(1) to[fore, black,->, cell=0.2, curve={ratio=0.20000000000000004}, ] node[coordinate](8){} (6) 
; 
\path[->] 
(1) edge["${\scriptstyle }$", pos=0.5, fore, black,->, cell=0.2, curve={ratio=0.20000000000000004}, ] (6) 
(0) edge["${\scriptstyle }$", pos=0.5, fore, black,->, ] (1) 
(1) edge["${\scriptstyle }$", pos=0.5, fore, black,->, ] (2) 
(0) edge["${\scriptstyle }$", pos=0.5, fore, black,->, ] (2) 
(1) edge["${\scriptstyle }$", pos=0.5, fore, black,->, cell=0.2, curve={ratio=-0.30000000000000004}, ] (6) 
; 
\end{tikzpicture}
% END OF GENERATED LATEX
\]
Alternatively, we could have a built-in mechanism to handle proof-irrelevant sorts.
\item[Associativity of composition]
We use the blue color for the specifying which elements 
are related by the equality tag.
\begin{equation}
    % YADE DIAGRAM rew-assoc.yade
    % GENERATED LATEX
    \begin{tikzpicture}[every node/.style={outer sep=0pt,anchor=base,text height=1.2ex, text depth=0.25ex}] 
\node[inner sep=5pt] (0) at (16.666666666666668em, -7.142857142857143em) {$\bullet$} ; 
\node[inner sep=5pt] (1) at (21.428571428571427em, -7.142857142857143em) {$\bullet$} ; 
\node[inner sep=5pt] (2) at (26.19047619047619em, -7.142857142857143em) {$\bullet$} ; 
\node[inner sep=5pt] (3) at (30.952380952380953em, -7.142857142857143em) {$\bullet$} ; 
\node[inner sep=5pt] (4) at (29.904761904761905em, -2.705357142857143em) {$\textcolor{red}{=}$} ; 
\path 
(0) to[fore, black,->, ] node[coordinate](5){} (1) 
(1) to[fore, black,->, ] node[coordinate](6){} (2) 
(2) to[fore, black,->, ] node[coordinate](7){} (3) 
(1) to[fore, black,->, curve={ratio=-0.5}, ] node[coordinate](8){} (3) 
(0) to[fore, blue,->, curve={ratio=-0.7999999999999999}, ] node[coordinate](9){} (3) 
(0) to[fore, black,->, curve={ratio=0.5}, ] node[coordinate](10){} (2) 
(0) to[fore, blue,->, curve={ratio=0.7999999999999999}, ] node[coordinate](11){} (3) 
(2) to[fore, black,->, cell=0.2, ] node[coordinate](12){} (8) 
(1) to[fore, black,->, cell=0.2, ] node[coordinate](13){} (9) 
(1) to[fore, black,->, cell=0.2, ] node[coordinate](14){} (10) 
(2) to[fore, black,->, cell=0.2, ] node[coordinate](15){} (11) 
; 
\path[->] 
(0) edge["${\scriptstyle }$", pos=0.5, fore, black,->, ] (1) 
(1) edge["${\scriptstyle }$", pos=0.5, fore, black,->, ] (2) 
(2) edge["${\scriptstyle }$", pos=0.5, fore, black,->, ] (3) 
(1) edge["${\scriptstyle }$", pos=0.5, fore, black,->, curve={ratio=-0.5}, ] (3) 
(0) edge["${\scriptstyle }$", pos=0.5, fore, blue,->, curve={ratio=-0.7999999999999999}, ] (3) 
(0) edge["${\scriptstyle }$", pos=0.5, fore, black,->, curve={ratio=0.5}, ] (2) 
(0) edge["${\scriptstyle }$", pos=0.5, fore, blue,->, curve={ratio=0.7999999999999999}, ] (3) 
(2) edge["${\scriptstyle }$", pos=0.5, fore, black,->, cell=0.2, ] (8) 
(1) edge["${\scriptstyle }$", pos=0.5, fore, black,->, cell=0.2, ] (9) 
(1) edge["${\scriptstyle }$", pos=0.5, fore, black,->, cell=0.2, ] (10) 
(2) edge["${\scriptstyle }$", pos=0.5, fore, black,->, cell=0.2, ] (11) 
; 
\end{tikzpicture}
    % END OF GENERATED LATEX
    \label{eq:rew-assoc}
    \tag{Assoc}
    \end{equation}
% \item[Chosen monormophisms]
% Remember from \Cref{ex:proof-mono} that our algebraic structure includes a choice of two composable monomorphisms, so that we have a rewriting rule creating two composable monomorphisms out of nowhere.
% \begin{equation}
% % YADE DIAGRAM nowhere-creation.yade
% GENERATED LATEX
% \begin{tikzpicture}[every node/.style={outer sep=0pt,anchor=base,text height=1.2ex, text depth=0.25ex}] 
\node[inner sep=5pt] (0) at (7.142857142857143em, -7.142857142857143em) {$\textcolor{red}{\bullet}$} ; 
\node[inner sep=5pt] (1) at (11.904761904761905em, -7.142857142857143em) {$\textcolor{red}{\bullet}$} ; 
\node[inner sep=5pt] (2) at (16.666666666666668em, -7.142857142857143em) {$\textcolor{red}{\bullet}$} ; 
\path 
(0) to[fore, red,->, into, ] node[coordinate](3){} (1) 
(1) to[fore, red,->, into, ] node[coordinate](4){} (2) 
; 
\path[->] 
(0) edge["${\scriptstyle }$", pos=0.5, fore, red,->, into, ] (1) 
(1) edge["${\scriptstyle }$", pos=0.5, fore, red,->, into, ] (2) 
; 
\end{tikzpicture}
% END OF GENERATED LATEX
% % GENERATED LATEX
% \begin{tikzpicture}[every node/.style={outer sep=0pt,anchor=base,text height=1.2ex, text depth=0.25ex}] 
\node[inner sep=5pt] (0) at (7.142857142857143em, -7.142857142857143em) {$\textcolor{red}{\bullet}$} ; 
\node[inner sep=5pt] (1) at (11.904761904761905em, -7.142857142857143em) {$\textcolor{red}{\bullet}$} ; 
\node[inner sep=5pt] (2) at (16.666666666666668em, -7.142857142857143em) {$\textcolor{red}{\bullet}$} ; 
\path 
(0) to[fore, red,->, into, ] node[coordinate](3){} (1) 
(1) to[fore, red,->, into, ] node[coordinate](4){} (2) 
; 
\path[->] 
(0) edge["${\scriptstyle }$", pos=0.5, fore, red,->, into, ] (1) 
(1) edge["${\scriptstyle }$", pos=0.5, fore, red,->, into, ] (2) 
; 
\end{tikzpicture}
% % END OF GENERATED LATEX
% \tag{Mono-Create}
% \label{eq:nowhere-creation}
% \end{equation}
\end{description}
\end{example}
\begin{question}
    What about identities? custom labelling of nodes?
\end{question}
% \paragraph{Creation of new admissible rewriting rules}
The user has two ways of creating new admissible rewriting rules, that is, rules that are valid in any model of the original set of rewrting rules.
\begin{description}
    \item[Chaining] Chaining rewriting rules;
    \item[Pruning] Removing distinguished elements from a rewriting diagram.
\end{description}
\paragraph{Example: the composition of two monomorphisms is monomorphic}
In this section, we show that the previous set of rewriting rules is enough to build the following rewriting rule, that shows that the composition of two monomorphisms is monomorphic.
\begin{align}
    % YADE DIAGRAM compo-mono-simplest-rule.yade
    % GENERATED LATEX
    \begin{tikzpicture}[every node/.style={outer sep=0pt,anchor=base,text height=1.2ex, text depth=0.25ex}] 
\node[inner sep=5pt] (0) at (21.428571428571427em, -11.904761904761905em) {$\textcolor{black}{\bullet}$} ; 
\node[inner sep=5pt] (1) at (26.19047619047619em, -11.904761904761905em) {$\textcolor{black}{\bullet}$} ; 
\node[inner sep=5pt] (2) at (30.952380952380953em, -11.904761904761905em) {$\textcolor{black}{\bullet}$} ; 
\node[inner sep=5pt] (3) at (16.666666666666668em, -11.904761904761905em) {$\bullet$} ; 
\node[inner sep=5pt] (4) at (18.904761904761905em, -11.848214285714286em) {$\textcolor{red}{=}$} ; 
\path 
(0) to[fore, black,->, into, ] node[coordinate](5){} (1) 
(1) to[fore, black,->, into, ] node[coordinate](6){} (2) 
(3) to[fore, black,->, curve={ratio=-0.2}, ] node[coordinate](7){} (0) 
(3) to[fore, black,->, curve={ratio=-0.6}, ] node[coordinate](8){} (2) 
(0) to[fore, black,->, curve={ratio=-0.6}, ] node[coordinate](9){} (2) 
(3) to[fore, black,->, curve={ratio=0.2}, ] node[coordinate](10){} (0) 
(1) to[fore, black,->, cell=0.2, curve={ratio=0.2}, ] node[coordinate](11){} (9) 
(0) to[fore, black,->, cell=0.2, curve={ratio=-0.4}, ] node[coordinate](12){} (8) 
(0) to[fore, black,->, cell=0.2, curve={ratio=0.1}, ] node[coordinate](13){} (8) 
; 
\path[->] 
(0) edge["${\scriptstyle }$", pos=0.5, fore, black,->, into, ] (1) 
(1) edge["${\scriptstyle }$", pos=0.5, fore, black,->, into, ] (2) 
(3) edge["${\scriptstyle }$", pos=0.5, fore, black,->, curve={ratio=-0.2}, ] (0) 
(3) edge["${\scriptstyle }$", pos=0.5, fore, black,->, curve={ratio=-0.6}, ] (2) 
(0) edge["${\scriptstyle }$", pos=0.5, fore, black,->, curve={ratio=-0.6}, ] (2) 
(3) edge["${\scriptstyle }$", pos=0.5, fore, black,->, curve={ratio=0.2}, ] (0) 
(1) edge["${\scriptstyle }$", pos=0.5, fore, black,->, cell=0.2, curve={ratio=0.2}, ] (9) 
(0) edge["${\scriptstyle }$", pos=0.5, fore, black,->, cell=0.2, curve={ratio=-0.4}, ] (8) 
(0) edge["${\scriptstyle }$", pos=0.5, fore, black,->, cell=0.2, curve={ratio=0.1}, ] (8) 
; 
\end{tikzpicture}
    % END OF GENERATED LATEX
    \label{eq:compo-mono-simplest-rule}
\end{align}

We start by chaining rewriting rules as follows.
We use blue to denote a match, and the red color 
to show the added part, and violet (or orange) for the match and added part.
\pagebreak
\begin{align*}
    \allowdisplaybreaks
 % YADE DIAGRAM compo-mono-ws.yade
 % GENERATED LATEX
 \begin{tikzpicture}[every node/.style={outer sep=0pt,anchor=base,text height=1.2ex, text depth=0.25ex}] 
\node[inner sep=5pt] (0) at (10.46875em, -1.1881510416666667em) {$ $} ; 
\node[inner sep=5pt] (1) at (13.020833333333334em, -7.8125em) {$\textcolor{black}{\bullet}$} ; 
\node[inner sep=5pt] (2) at (18.229166666666668em, -7.8125em) {$\textcolor{black}{\bullet}$} ; 
\node[inner sep=5pt] (3) at (23.4375em, -7.8125em) {$\textcolor{black}{\bullet}$} ; 
\node[inner sep=5pt] (4) at (25.963541666666668em, -2.594401041666667em) {$ $} ; 
\node[inner sep=5pt] (5) at (26.015625em, -11.240234375em) {$\eqref{eq:rew-comp}$} ; 
\node[inner sep=5pt] (6) at (7.8125em, -7.8125em) {$\bullet$} ; 
\node[inner sep=5pt] (7) at (33.85416666666667em, -7.8125em) {$\textcolor{black}{\bullet}$} ; 
\node[inner sep=5pt] (8) at (39.0625em, -7.8125em) {$\textcolor{black}{\bullet}$} ; 
\node[inner sep=5pt] (9) at (44.270833333333336em, -7.8125em) {$\textcolor{black}{\bullet}$} ; 
\node[inner sep=5pt] (10) at (28.645833333333336em, -7.8125em) {$\bullet$} ; 
\node[inner sep=5pt] (11) at (46.848958333333336em, -2.906901041666667em) {$ $} ; 
\node[inner sep=5pt] (12) at (46.90104166666667em, -10.667317708333334em) {$\eqref{eq:rew-comp}$} ; 
\node[inner sep=5pt] (13) at (13.020833333333334em, -16.614583333333336em) {$\textcolor{black}{\bullet}$} ; 
\node[inner sep=5pt] (14) at (18.229166666666668em, -16.614583333333336em) {$\textcolor{black}{\bullet}$} ; 
\node[inner sep=5pt] (15) at (23.4375em, -16.614583333333336em) {$\textcolor{black}{\bullet}$} ; 
\node[inner sep=5pt] (16) at (7.8125em, -16.614583333333336em) {$\bullet$} ; 
\node[inner sep=5pt] (17) at (25.963541666666668em, -11.813151041666668em) {$ $} ; 
\node[inner sep=5pt] (18) at (26.015625em, -20.458984375em) {$\eqref{eq:rew-comp}$} ; 
\node[inner sep=5pt] (19) at (33.85416666666667em, -16.614583333333336em) {$\textcolor{black}{\bullet}$} ; 
\node[inner sep=5pt] (20) at (39.0625em, -16.614583333333336em) {$\textcolor{black}{\bullet}$} ; 
\node[inner sep=5pt] (21) at (44.270833333333336em, -16.614583333333336em) {$\textcolor{black}{\bullet}$} ; 
\node[inner sep=5pt] (22) at (28.645833333333336em, -16.614583333333336em) {$\bullet$} ; 
\node[inner sep=5pt] (23) at (46.953125em, -11.136067708333334em) {$ $} ; 
\node[inner sep=5pt] (24) at (47.005208333333336em, -19.781901041666668em) {$\eqref{eq:rew-comp}$} ; 
\node[inner sep=5pt] (25) at (13.020833333333334em, -29.791666666666668em) {$\textcolor{black}{\bullet}$} ; 
\node[inner sep=5pt] (26) at (18.229166666666668em, -29.791666666666668em) {$\textcolor{black}{\bullet}$} ; 
\node[inner sep=5pt] (27) at (23.4375em, -29.791666666666668em) {$\textcolor{black}{\bullet}$} ; 
\node[inner sep=5pt] (28) at (7.8125em, -29.791666666666668em) {$\bullet$} ; 
\node[inner sep=5pt] (29) at (25.911458333333336em, -26.604817708333336em) {$ $} ; 
\node[inner sep=5pt] (30) at (25.963541666666668em, -35.25065104166667em) {$\eqref{eq:rew-assoc}$} ; 
\node[inner sep=5pt] (31) at (33.85416666666667em, -29.791666666666668em) {$\textcolor{black}{\bullet}$} ; 
\node[inner sep=5pt] (32) at (39.0625em, -29.791666666666668em) {$\textcolor{black}{\bullet}$} ; 
\node[inner sep=5pt] (33) at (44.270833333333336em, -29.791666666666668em) {$\textcolor{black}{\bullet}$} ; 
\node[inner sep=5pt] (34) at (28.645833333333336em, -29.791666666666668em) {$\bullet$} ; 
\node[inner sep=5pt] (35) at (46.848958333333336em, -26.448567708333336em) {$ $} ; 
\node[inner sep=5pt] (36) at (46.90104166666667em, -35.09440104166667em) {$\eqref{eq:rew-assoc}$} ; 
\node[inner sep=5pt] (37) at (13.020833333333334em, -45.41666666666667em) {$\textcolor{black}{\bullet}$} ; 
\node[inner sep=5pt] (38) at (18.229166666666668em, -45.41666666666667em) {$\textcolor{black}{\bullet}$} ; 
\node[inner sep=5pt] (39) at (23.4375em, -45.41666666666667em) {$\textcolor{black}{\bullet}$} ; 
\node[inner sep=5pt] (40) at (7.8125em, -45.41666666666667em) {$\bullet$} ; 
\node[inner sep=5pt] (41) at (26.119791666666668em, -40.87565104166667em) {$ $} ; 
\node[inner sep=5pt] (42) at (26.171875em, -49.521484375em) {$\eqref{eq:rew-mono-fo}$} ; 
\node[inner sep=5pt] (43) at (33.85416666666667em, -45.41666666666667em) {$\textcolor{black}{\bullet}$} ; 
\node[inner sep=5pt] (44) at (39.0625em, -45.41666666666667em) {$\textcolor{black}{\bullet}$} ; 
\node[inner sep=5pt] (45) at (44.270833333333336em, -45.41666666666667em) {$\textcolor{black}{\bullet}$} ; 
\node[inner sep=5pt] (46) at (28.645833333333336em, -45.41666666666667em) {$\bullet$} ; 
\node[inner sep=5pt] (47) at (46.90104166666667em, -40.458984375em) {$ $} ; 
\node[inner sep=5pt] (48) at (46.953125em, -49.104817708333336em) {$\eqref{eq:rew-mono-fo}$} ; 
\node[inner sep=5pt] (49) at (13.020833333333334em, -55.833333333333336em) {$\textcolor{black}{\bullet}$} ; 
\node[inner sep=5pt] (50) at (18.229166666666668em, -55.833333333333336em) {$\textcolor{black}{\bullet}$} ; 
\node[inner sep=5pt] (51) at (23.4375em, -55.833333333333336em) {$\textcolor{black}{\bullet}$} ; 
\node[inner sep=5pt] (52) at (7.8125em, -55.833333333333336em) {$\bullet$} ; 
\path 
(1) to[fore, blue,->, into, ] node[coordinate](53){} (2) 
(2) to[fore, black,->, into, ] node[coordinate](54){} (3) 
(4) to[fore, black,-,] node[coordinate](55){} (5) 
(6) to[fore, blue,->, curve={ratio=0.2}, ] node[coordinate](56){} (1) 
(6) to[fore, black,->, curve={ratio=-0.30000000000000004}, ] node[coordinate](57){} (1) 
(6) to[fore, black,->, curve={ratio=-0.6}, ] node[coordinate](58){} (3) 
(1) to[fore, black,->, curve={ratio=-0.5}, ] node[coordinate](59){} (3) 
(7) to[fore, blue,->, into, ] node[coordinate](60){} (8) 
(8) to[fore, black,->, into, ] node[coordinate](61){} (9) 
(10) to[fore, black,->, curve={ratio=0.2}, ] node[coordinate](62){} (7) 
(10) to[fore, blue,->, curve={ratio=-0.30000000000000004}, ] node[coordinate](63){} (7) 
(10) to[fore, black,->, curve={ratio=-0.6}, ] node[coordinate](64){} (9) 
(7) to[fore, black,->, curve={ratio=-0.5}, ] node[coordinate](65){} (9) 
(10) to[fore, red,->, curve={ratio=0.6}, ] node[coordinate](66){} (8) 
(11) to[fore, black,-,] node[coordinate](67){} (12) 
(13) to[fore, black,->, into, ] node[coordinate](68){} (14) 
(14) to[fore, blue,->, into, ] node[coordinate](69){} (15) 
(16) to[fore, black,->, curve={ratio=0.2}, ] node[coordinate](70){} (13) 
(16) to[fore, black,->, curve={ratio=-0.30000000000000004}, ] node[coordinate](71){} (13) 
(16) to[fore, black,->, curve={ratio=-0.6}, ] node[coordinate](72){} (15) 
(13) to[fore, black,->, curve={ratio=-0.5}, ] node[coordinate](73){} (15) 
(16) to[fore, blue,->, curve={ratio=0.6}, ] node[coordinate](74){} (14) 
(16) to[fore, red,->, curve={ratio=0.8999999999999999}, ] node[coordinate](75){} (14) 
(17) to[fore, black,-,] node[coordinate](76){} (18) 
(19) to[fore, black,->, into, ] node[coordinate](77){} (20) 
(20) to[fore, blue,->, into, ] node[coordinate](78){} (21) 
(22) to[fore, black,->, curve={ratio=0.2}, ] node[coordinate](79){} (19) 
(22) to[fore, black,->, curve={ratio=-0.30000000000000004}, ] node[coordinate](80){} (19) 
(22) to[fore, black,->, curve={ratio=-0.6}, ] node[coordinate](81){} (21) 
(19) to[fore, black,->, curve={ratio=-0.5}, ] node[coordinate](82){} (21) 
(22) to[fore, black,->, curve={ratio=0.6}, ] node[coordinate](83){} (20) 
(22) to[fore, blue,->, curve={ratio=0.8999999999999999}, ] node[coordinate](84){} (20) 
(22) to[fore, red,->, curve={ratio=0.8999999999999999}, ] node[coordinate](85){} (21) 
(23) to[fore, black,-,] node[coordinate](86){} (24) 
(25) to[fore, blue,->, into, ] node[coordinate](87){} (26) 
(26) to[fore, blue,->, into, ] node[coordinate](88){} (27) 
(28) to[fore, black,->, curve={ratio=0.2}, ] node[coordinate](89){} (25) 
(28) to[fore, blue,->, curve={ratio=-0.30000000000000004}, ] node[coordinate](90){} (25) 
(28) to[fore, blue,->, curve={ratio=-0.6}, ] node[coordinate](91){} (27) 
(25) to[fore, blue,->, curve={ratio=-0.5}, ] node[coordinate](92){} (27) 
(28) to[fore, black,->, curve={ratio=0.6}, ] node[coordinate](93){} (26) 
(28) to[fore, blue,->, curve={ratio=0.7999999999999999}, ] node[coordinate](94){} (26) 
(28) to[fore, red,->, curve={ratio=1.2}, ] node[coordinate](95){} (27) 
(28) to[fore, blue,->, curve={ratio=0.8999999999999999}, ] node[coordinate](96){} (27) 
(29) to[fore, black,-,] node[coordinate](97){} (30) 
(31) to[fore, blue,->, into, ] node[coordinate](98){} (32) 
(32) to[fore, blue,->, into, ] node[coordinate](99){} (33) 
(34) to[fore, blue,->, curve={ratio=0.2}, ] node[coordinate](100){} (31) 
(34) to[fore, black,->, curve={ratio=-0.30000000000000004}, ] node[coordinate](101){} (31) 
(34) to[fore, blue,->, curve={ratio=-0.6}, ] node[coordinate](102){} (33) 
(31) to[fore, blue,->, curve={ratio=-0.5}, ] node[coordinate](103){} (33) 
(34) to[fore, blue,->, curve={ratio=0.6}, ] node[coordinate](104){} (32) 
(34) to[fore, black,->, curve={ratio=0.7999999999999999}, ] node[coordinate](105){} (32) 
(34) to[fore, blue,->, curve={ratio=0.8999999999999999}, ] node[coordinate](106){} (33) 
(35) to[fore, black,-,] node[coordinate](107){} (36) 
(37) to[fore, black,->, into, ] node[coordinate](108){} (38) 
(38) to[fore, blue,->, into, ] node[coordinate](109){} (39) 
(40) to[fore, black,->, curve={ratio=0.2}, ] node[coordinate](110){} (37) 
(40) to[fore, black,->, curve={ratio=-0.30000000000000004}, ] node[coordinate](111){} (37) 
(40) to[fore, blue,->, curve={ratio=-0.6}, ] node[coordinate](112){} (39) 
(37) to[fore, black,->, curve={ratio=-0.6}, ] node[coordinate](113){} (39) 
(40) to[fore, blue,->, curve={ratio=0.6}, ] node[coordinate](114){} (38) 
(40) to[fore, blue,->, curve={ratio=0.7999999999999999}, ] node[coordinate](115){} (38) 
(41) to[fore, black,-,] node[coordinate](116){} (42) 
(43) to[fore, blue,->, into, ] node[coordinate](117){} (44) 
(44) to[fore, black,->, into, ] node[coordinate](118){} (45) 
(46) to[fore, blue,->, curve={ratio=0.2}, ] node[coordinate](119){} (43) 
(46) to[fore, blue,->, curve={ratio=-0.30000000000000004}, ] node[coordinate](120){} (43) 
(46) to[fore, black,->, curve={ratio=-0.6}, ] node[coordinate](121){} (45) 
(43) to[fore, black,->, curve={ratio=-0.6}, ] node[coordinate](122){} (45) 
(46) to[fore, blue,->, curve={ratio=0.6}, ] node[coordinate](123){} (44) 
(47) to[fore, black,-,] node[coordinate](124){} (48) 
(49) to[fore, black,->, into, ] node[coordinate](125){} (50) 
(50) to[fore, black,->, into, ] node[coordinate](126){} (51) 
(52) to[fore, black,->, ] node[coordinate](127){} (49) 
(52) to[fore, black,->, curve={ratio=-0.6}, ] node[coordinate](128){} (51) 
(49) to[fore, black,->, curve={ratio=-0.6}, ] node[coordinate](129){} (51) 
(52) to[fore, black,->, curve={ratio=0.6}, ] node[coordinate](130){} (50) 
(2) to[fore, black,->, cell=0.05, ] node[coordinate](131){} (59) 
(1) to[fore, black,->, cell=0.05, curve={ratio=-0.4}, ] node[coordinate](132){} (58) 
(1) to[fore, black,->, cell=0.05, curve={ratio=0.1}, ] node[coordinate](133){} (58) 
(8) to[fore, black,->, cell=0.05, ] node[coordinate](134){} (65) 
(7) to[fore, black,->, cell=0.05, curve={ratio=-0.4}, ] node[coordinate](135){} (64) 
(7) to[fore, black,->, cell=0.05, curve={ratio=0.1}, ] node[coordinate](136){} (64) 
(7) to[fore, red,->, cell=0.05, ] node[coordinate](137){} (66) 
(14) to[fore, black,->, cell=0.05, ] node[coordinate](138){} (73) 
(13) to[fore, black,->, cell=0.05, curve={ratio=-0.4}, ] node[coordinate](139){} (72) 
(13) to[fore, black,->, cell=0.05, curve={ratio=0.1}, ] node[coordinate](140){} (72) 
(13) to[fore, black,->, cell=0.05, curve={ratio=-0.2}, ] node[coordinate](141){} (74) 
(13) to[fore, red,->, cell=0.05, curve={ratio=0.4}, ] node[coordinate](142){} (75) 
(20) to[fore, black,->, cell=0.05, ] node[coordinate](143){} (82) 
(19) to[fore, black,->, cell=0.05, curve={ratio=-0.4}, ] node[coordinate](144){} (81) 
(19) to[fore, black,->, cell=0.05, curve={ratio=0.1}, ] node[coordinate](145){} (81) 
(19) to[fore, black,->, cell=0.05, curve={ratio=-0.2}, ] node[coordinate](146){} (83) 
(19) to[fore, black,->, cell=0.05, curve={ratio=0.4}, ] node[coordinate](147){} (84) 
(20) to[fore, red,->, cell=0.05, ] node[coordinate](148){} (85) 
(26) to[fore, blue,->, cell=0.05, ] node[coordinate](149){} (92) 
(25) to[fore, blue,->, cell=0.05, curve={ratio=-0.4}, ] node[coordinate](150){} (91) 
(25) to[fore, black,->, cell=0.05, curve={ratio=0.1}, ] node[coordinate](151){} (91) 
(25) to[fore, black,->, cell=0.05, curve={ratio=-0.2}, ] node[coordinate](152){} (93) 
(25) to[fore, blue,->, cell=0.05, curve={ratio=0.4}, ] node[coordinate](153){} (94) 
(26) to[fore, red,->, cell=0.05, curve={ratio=-0.30000000000000004}, ] node[coordinate](154){} (95) 
(26) to[fore, blue,->, cell=0.05, ] node[coordinate](155){} (96) 
(32) to[fore, blue,->, cell=0.05, ] node[coordinate](156){} (103) 
(31) to[fore, black,->, cell=0.05, curve={ratio=-0.4}, ] node[coordinate](157){} (102) 
(31) to[fore, blue,->, cell=0.05, curve={ratio=0.1}, ] node[coordinate](158){} (102) 
(31) to[fore, blue,->, cell=0.05, curve={ratio=-0.2}, ] node[coordinate](159){} (104) 
(31) to[fore, black,->, cell=0.05, curve={ratio=0.4}, ] node[coordinate](160){} (105) 
(32) to[fore, blue,->, cell=0.05, curve={ratio=-0.30000000000000004}, ] node[coordinate](161){} (106) 
(32) to[fore, black,->, cell=0.05, ] node[coordinate](162){} (102) 
(38) to[fore, black,->, cell=0.05, curve={ratio=0.2}, ] node[coordinate](163){} (113) 
(37) to[fore, black,->, cell=0.05, curve={ratio=-0.4}, ] node[coordinate](164){} (112) 
(37) to[fore, black,->, cell=0.05, curve={ratio=0.1}, ] node[coordinate](165){} (112) 
(37) to[fore, black,->, cell=0.05, curve={ratio=-0.2}, ] node[coordinate](166){} (114) 
(37) to[fore, black,->, cell=0.05, curve={ratio=0.4}, ] node[coordinate](167){} (115) 
(38) to[fore, blue,->, cell=0.05, curve={ratio=-0.30000000000000004}, ] node[coordinate](168){} (112) 
(38) to[fore, blue,->, cell=0.05, curve={ratio=0.20000000000000004}, ] node[coordinate](169){} (112) 
(44) to[fore, black,->, cell=0.05, curve={ratio=0.2}, ] node[coordinate](170){} (122) 
(43) to[fore, black,->, cell=0.05, curve={ratio=-0.4}, ] node[coordinate](171){} (121) 
(43) to[fore, black,->, cell=0.05, curve={ratio=0.1}, ] node[coordinate](172){} (121) 
(43) to[fore, blue,->, cell=0.05, curve={ratio=-0.2}, ] node[coordinate](173){} (123) 
(44) to[fore, black,->, cell=0.05, curve={ratio=-0.2}, ] node[coordinate](174){} (121) 
(43) to[fore, blue,->, cell=0.05, curve={ratio=0.4}, ] node[coordinate](175){} (123) 
(44) to[fore, black,->, cell=0.05, curve={ratio=0.2}, ] node[coordinate](176){} (121) 
(50) to[fore, black,->, cell=0.05, curve={ratio=0.2}, ] node[coordinate](177){} (129) 
(49) to[fore, black,->, cell=0.05, curve={ratio=-0.4}, ] node[coordinate](178){} (128) 
(49) to[fore, black,->, cell=0.05, curve={ratio=0.1}, ] node[coordinate](179){} (128) 
(49) to[fore, black,->, cell=0.05, curve={ratio=-0.2}, ] node[coordinate](180){} (130) 
(50) to[fore, black,->, cell=0.05, curve={ratio=-0.2}, ] node[coordinate](181){} (128) 
(49) to[fore, black,->, cell=0.05, curve={ratio=0.4}, ] node[coordinate](182){} (130) 
(50) to[fore, black,->, cell=0.05, curve={ratio=0.2}, ] node[coordinate](183){} (128) 
; 
\path[->, transform shape, every edge quotes/.style={}, scale=0.7] 
(1) edge["${ }$" auto=left, pos=0.5, fore, blue,->, into, ] (2) 
(2) edge["${ }$" auto=left, pos=0.5, fore, black,->, into, ] (3) 
(4) edge["${ }$" auto=left, pos=0.5, fore, black,-,] (5) 
(6) edge["${ }$" auto=left, pos=0.5, fore, blue,->, curve={ratio=0.2}, ] (1) 
(6) edge["${ }$" auto=left, pos=0.5, fore, black,->, curve={ratio=-0.30000000000000004}, ] (1) 
(6) edge["${ }$" auto=left, pos=0.5, fore, black,->, curve={ratio=-0.6}, ] (3) 
(1) edge["${ }$" auto=left, pos=0.5, fore, black,->, curve={ratio=-0.5}, ] (3) 
(7) edge["${ }$" auto=left, pos=0.5, fore, blue,->, into, ] (8) 
(8) edge["${ }$" auto=left, pos=0.5, fore, black,->, into, ] (9) 
(10) edge["${ }$" auto=left, pos=0.5, fore, black,->, curve={ratio=0.2}, ] (7) 
(10) edge["${ }$" auto=left, pos=0.5, fore, blue,->, curve={ratio=-0.30000000000000004}, ] (7) 
(10) edge["${ }$" auto=left, pos=0.5, fore, black,->, curve={ratio=-0.6}, ] (9) 
(7) edge["${ }$" auto=left, pos=0.5, fore, black,->, curve={ratio=-0.5}, ] (9) 
(10) edge["${ }$" auto=right, pos=0.5, fore, red,->, curve={ratio=0.6}, ] (8) 
(11) edge["${ }$" auto=left, pos=0.5, fore, black,-,] (12) 
(13) edge["${ }$" auto=left, pos=0.5, fore, black,->, into, ] (14) 
(14) edge["${ }$" auto=left, pos=0.5, fore, blue,->, into, ] (15) 
(16) edge["${ }$" auto=left, pos=0.5, fore, black,->, curve={ratio=0.2}, ] (13) 
(16) edge["${ }$" auto=left, pos=0.5, fore, black,->, curve={ratio=-0.30000000000000004}, ] (13) 
(16) edge["${ }$" auto=left, pos=0.5, fore, black,->, curve={ratio=-0.6}, ] (15) 
(13) edge["${ }$" auto=left, pos=0.5, fore, black,->, curve={ratio=-0.5}, ] (15) 
(16) edge["${ }$" auto=right, pos=0.5, fore, blue,->, curve={ratio=0.6}, ] (14) 
(16) edge["${ }$" auto=left, pos=0.5, fore, red,->, curve={ratio=0.8999999999999999}, ] (14) 
(17) edge["${ }$" auto=left, pos=0.5, fore, black,-,] (18) 
(19) edge["${ }$" auto=left, pos=0.5, fore, black,->, into, ] (20) 
(20) edge["${ }$" auto=left, pos=0.5, fore, blue,->, into, ] (21) 
(22) edge["${ }$" auto=left, pos=0.5, fore, black,->, curve={ratio=0.2}, ] (19) 
(22) edge["${ }$" auto=left, pos=0.5, fore, black,->, curve={ratio=-0.30000000000000004}, ] (19) 
(22) edge["${ }$" auto=left, pos=0.5, fore, black,->, curve={ratio=-0.6}, ] (21) 
(19) edge["${ }$" auto=left, pos=0.5, fore, black,->, curve={ratio=-0.5}, ] (21) 
(22) edge["${ }$" auto=right, pos=0.5, fore, black,->, curve={ratio=0.6}, ] (20) 
(22) edge["${ }$" auto=left, pos=0.5, fore, blue,->, curve={ratio=0.8999999999999999}, ] (20) 
(22) edge["${ }$" auto=right, pos=0.5, fore, red,->, curve={ratio=0.8999999999999999}, ] (21) 
(23) edge["${ }$" auto=left, pos=0.5, fore, black,-,] (24) 
(25) edge["${ }$" auto=left, pos=0.5, fore, blue,->, into, ] (26) 
(26) edge["${ }$" auto=left, pos=0.5, fore, blue,->, into, ] (27) 
(28) edge["${ }$" auto=left, pos=0.5, fore, black,->, curve={ratio=0.2}, ] (25) 
(28) edge["${ }$" auto=left, pos=0.5, fore, blue,->, curve={ratio=-0.30000000000000004}, ] (25) 
(28) edge["${ }$" auto=left, pos=0.5, fore, blue,->, curve={ratio=-0.6}, ] (27) 
(25) edge["${ }$" auto=left, pos=0.5, fore, blue,->, curve={ratio=-0.5}, ] (27) 
(28) edge["${ }$" auto=right, pos=0.5, fore, black,->, curve={ratio=0.6}, ] (26) 
(28) edge["${ }$" auto=left, pos=0.5, fore, blue,->, curve={ratio=0.7999999999999999}, ] (26) 
(28) edge["${ }$" auto=right, pos=0.5, fore, red,->, curve={ratio=1.2}, ] (27) 
(28) edge["${ }$" auto=left, pos=0.5, fore, blue,->, curve={ratio=0.8999999999999999}, ] (27) 
(29) edge["${ }$" auto=left, pos=0.5, fore, black,-,] (30) 
(31) edge["${ }$" auto=left, pos=0.5, fore, blue,->, into, ] (32) 
(32) edge["${ }$" auto=left, pos=0.5, fore, blue,->, into, ] (33) 
(34) edge["${ }$" auto=left, pos=0.5, fore, blue,->, curve={ratio=0.2}, ] (31) 
(34) edge["${ }$" auto=left, pos=0.5, fore, black,->, curve={ratio=-0.30000000000000004}, ] (31) 
(34) edge["${ }$" auto=left, pos=0.5, fore, blue,->, curve={ratio=-0.6}, ] (33) 
(31) edge["${ }$" auto=left, pos=0.5, fore, blue,->, curve={ratio=-0.5}, ] (33) 
(34) edge["${ }$" auto=right, pos=0.5, fore, blue,->, curve={ratio=0.6}, ] (32) 
(34) edge["${ }$" auto=left, pos=0.5, fore, black,->, curve={ratio=0.7999999999999999}, ] (32) 
(34) edge["${ }$" auto=right, pos=0.5, fore, blue,->, curve={ratio=0.8999999999999999}, ] (33) 
(35) edge["${ }$" auto=left, pos=0.5, fore, black,-,] (36) 
(37) edge["${ }$" auto=left, pos=0.5, fore, black,->, into, ] (38) 
(38) edge["${ }$" auto=left, pos=0.5, fore, blue,->, into, ] (39) 
(40) edge["${ }$" auto=left, pos=0.5, fore, black,->, curve={ratio=0.2}, ] (37) 
(40) edge["${ }$" auto=left, pos=0.5, fore, black,->, curve={ratio=-0.30000000000000004}, ] (37) 
(40) edge["${ }$" auto=left, pos=0.5, fore, blue,->, curve={ratio=-0.6}, ] (39) 
(37) edge["${ }$" auto=left, pos=0.5, fore, black,->, curve={ratio=-0.6}, ] (39) 
(40) edge["${ }$" auto=right, pos=0.5, fore, blue,->, curve={ratio=0.6}, ] (38) 
(40) edge["${ }$" auto=left, pos=0.5, fore, blue,->, curve={ratio=0.7999999999999999}, ] (38) 
(41) edge["${ }$" auto=left, pos=0.5, fore, black,-,] (42) 
(43) edge["${ }$" auto=left, pos=0.5, fore, blue,->, into, ] (44) 
(44) edge["${ }$" auto=left, pos=0.5, fore, black,->, into, ] (45) 
(46) edge["${ }$" auto=left, pos=0.5, fore, blue,->, curve={ratio=0.2}, ] (43) 
(46) edge["${ }$" auto=left, pos=0.5, fore, blue,->, curve={ratio=-0.30000000000000004}, ] (43) 
(46) edge["${ }$" auto=left, pos=0.5, fore, black,->, curve={ratio=-0.6}, ] (45) 
(43) edge["${ }$" auto=left, pos=0.5, fore, black,->, curve={ratio=-0.6}, ] (45) 
(46) edge["${ }$" auto=right, pos=0.5, fore, blue,->, curve={ratio=0.6}, ] (44) 
(47) edge["${ }$" auto=left, pos=0.5, fore, black,-,] (48) 
(49) edge["${ }$" auto=left, pos=0.5, fore, black,->, into, ] (50) 
(50) edge["${ }$" auto=left, pos=0.5, fore, black,->, into, ] (51) 
(52) edge["${ }$" auto=left, pos=0.5, fore, black,->, ] (49) 
(52) edge["${ }$" auto=left, pos=0.5, fore, black,->, curve={ratio=-0.6}, ] (51) 
(49) edge["${ }$" auto=left, pos=0.5, fore, black,->, curve={ratio=-0.6}, ] (51) 
(52) edge["${ }$" auto=right, pos=0.5, fore, black,->, curve={ratio=0.6}, ] (50) 
(2) edge["${ }$" auto=left, pos=0.5, fore, black,->, cell=0.05, ] (59) 
(1) edge["${ }$" auto=left, pos=0.5, fore, black,->, cell=0.05, curve={ratio=-0.4}, ] (58) 
(1) edge["${ }$" auto=left, pos=0.5, fore, black,->, cell=0.05, curve={ratio=0.1}, ] (58) 
(8) edge["${ }$" auto=left, pos=0.5, fore, black,->, cell=0.05, ] (65) 
(7) edge["${ }$" auto=left, pos=0.5, fore, black,->, cell=0.05, curve={ratio=-0.4}, ] (64) 
(7) edge["${ }$" auto=left, pos=0.5, fore, black,->, cell=0.05, curve={ratio=0.1}, ] (64) 
(7) edge["${ }$" auto=left, pos=0.5, fore, red,->, cell=0.05, ] (66) 
(14) edge["${ }$" auto=left, pos=0.5, fore, black,->, cell=0.05, ] (73) 
(13) edge["${ }$" auto=left, pos=0.5, fore, black,->, cell=0.05, curve={ratio=-0.4}, ] (72) 
(13) edge["${ }$" auto=left, pos=0.5, fore, black,->, cell=0.05, curve={ratio=0.1}, ] (72) 
(13) edge["${ }$" auto=left, pos=0.5, fore, black,->, cell=0.05, curve={ratio=-0.2}, ] (74) 
(13) edge["${ }$" auto=left, pos=0.5, fore, red,->, cell=0.05, curve={ratio=0.4}, ] (75) 
(20) edge["${ }$" auto=left, pos=0.5, fore, black,->, cell=0.05, ] (82) 
(19) edge["${ }$" auto=left, pos=0.5, fore, black,->, cell=0.05, curve={ratio=-0.4}, ] (81) 
(19) edge["${ }$" auto=left, pos=0.5, fore, black,->, cell=0.05, curve={ratio=0.1}, ] (81) 
(19) edge["${ }$" auto=left, pos=0.5, fore, black,->, cell=0.05, curve={ratio=-0.2}, ] (83) 
(19) edge["${ }$" auto=left, pos=0.5, fore, black,->, cell=0.05, curve={ratio=0.4}, ] (84) 
(20) edge["${ }$" auto=left, pos=0.5, fore, red,->, cell=0.05, ] (85) 
(26) edge["${ }$" auto=left, pos=0.5, fore, blue,->, cell=0.05, ] (92) 
(25) edge["${ }$" auto=left, pos=0.5, fore, blue,->, cell=0.05, curve={ratio=-0.4}, ] (91) 
(25) edge["${ }$" auto=left, pos=0.5, fore, black,->, cell=0.05, curve={ratio=0.1}, ] (91) 
(25) edge["${ }$" auto=left, pos=0.5, fore, black,->, cell=0.05, curve={ratio=-0.2}, ] (93) 
(25) edge["${ }$" auto=left, pos=0.5, fore, blue,->, cell=0.05, curve={ratio=0.4}, ] (94) 
(26) edge["${ }$" auto=left, pos=0.5, fore, red,->, cell=0.05, curve={ratio=-0.30000000000000004}, ] (95) 
(26) edge["${ }$" auto=left, pos=0.5, fore, blue,->, cell=0.05, ] (96) 
(32) edge["${ }$" auto=left, pos=0.5, fore, blue,->, cell=0.05, ] (103) 
(31) edge["${ }$" auto=left, pos=0.5, fore, black,->, cell=0.05, curve={ratio=-0.4}, ] (102) 
(31) edge["${ }$" auto=left, pos=0.5, fore, blue,->, cell=0.05, curve={ratio=0.1}, ] (102) 
(31) edge["${ }$" auto=left, pos=0.5, fore, blue,->, cell=0.05, curve={ratio=-0.2}, ] (104) 
(31) edge["${ }$" auto=left, pos=0.5, fore, black,->, cell=0.05, curve={ratio=0.4}, ] (105) 
(32) edge["${ }$" auto=left, pos=0.5, fore, blue,->, cell=0.05, curve={ratio=-0.30000000000000004}, ] (106) 
(32) edge["${ }$" auto=left, pos=0.5, fore, black,->, cell=0.05, ] (102) 
(38) edge["${ }$" auto=left, pos=0.5, fore, black,->, cell=0.05, curve={ratio=0.2}, ] (113) 
(37) edge["${ }$" auto=left, pos=0.5, fore, black,->, cell=0.05, curve={ratio=-0.4}, ] (112) 
(37) edge["${ }$" auto=left, pos=0.5, fore, black,->, cell=0.05, curve={ratio=0.1}, ] (112) 
(37) edge["${ }$" auto=left, pos=0.5, fore, black,->, cell=0.05, curve={ratio=-0.2}, ] (114) 
(37) edge["${ }$" auto=left, pos=0.5, fore, black,->, cell=0.05, curve={ratio=0.4}, ] (115) 
(38) edge["${ }$" auto=left, pos=0.5, fore, blue,->, cell=0.05, curve={ratio=-0.30000000000000004}, ] (112) 
(38) edge["${ }$" auto=left, pos=0.5, fore, blue,->, cell=0.05, curve={ratio=0.20000000000000004}, ] (112) 
(44) edge["${ }$" auto=left, pos=0.5, fore, black,->, cell=0.05, curve={ratio=0.2}, ] (122) 
(43) edge["${ }$" auto=left, pos=0.5, fore, black,->, cell=0.05, curve={ratio=-0.4}, ] (121) 
(43) edge["${ }$" auto=left, pos=0.5, fore, black,->, cell=0.05, curve={ratio=0.1}, ] (121) 
(43) edge["${ }$" auto=left, pos=0.5, fore, blue,->, cell=0.05, curve={ratio=-0.2}, ] (123) 
(44) edge["${ }$" auto=left, pos=0.5, fore, black,->, cell=0.05, curve={ratio=-0.2}, ] (121) 
(43) edge["${ }$" auto=left, pos=0.5, fore, blue,->, cell=0.05, curve={ratio=0.4}, ] (123) 
(44) edge["${ }$" auto=left, pos=0.5, fore, black,->, cell=0.05, curve={ratio=0.2}, ] (121) 
(50) edge["${ }$" auto=left, pos=0.5, fore, black,->, cell=0.05, curve={ratio=0.2}, ] (129) 
(49) edge["${ }$" auto=left, pos=0.5, fore, black,->, cell=0.05, curve={ratio=-0.4}, ] (128) 
(49) edge["${ }$" auto=left, pos=0.5, fore, black,->, cell=0.05, curve={ratio=0.1}, ] (128) 
(49) edge["${ }$" auto=left, pos=0.5, fore, black,->, cell=0.05, curve={ratio=-0.2}, ] (130) 
(50) edge["${ }$" auto=left, pos=0.5, fore, black,->, cell=0.05, curve={ratio=-0.2}, ] (128) 
(49) edge["${ }$" auto=left, pos=0.5, fore, black,->, cell=0.05, curve={ratio=0.4}, ] (130) 
(50) edge["${ }$" auto=left, pos=0.5, fore, black,->, cell=0.05, curve={ratio=0.2}, ] (128) 
; 
\end{tikzpicture}
 % END OF GENERATED LATEX
\end{align*}
\begin{question}
    Clearly, this is too complicated... How can we make it simpler?
    See~\Cref{app:simplified-proof} for some attempts.
    We should offer a way to focus on some subpart of the diagram (for example, elements that depends on some selected parts), hide some stuff.
\end{question}
This sequence is compactified by the software into a single rewriting rule:
\begin{align}
    % YADE DIAGRAM compo-mono-simple-rule.yade
    % GENERATED LATEX
    \begin{tikzpicture}[every node/.style={outer sep=0pt,anchor=base,text height=1.2ex, text depth=0.25ex}] 
\node[inner sep=5pt] (0) at (16.666666666666668em, -21.428571428571427em) {$\textcolor{black}{\bullet}$} ; 
\node[inner sep=5pt] (1) at (21.428571428571427em, -21.428571428571427em) {$\textcolor{black}{\bullet}$} ; 
\node[inner sep=5pt] (2) at (26.19047619047619em, -21.428571428571427em) {$\textcolor{black}{\bullet}$} ; 
\node[inner sep=5pt] (3) at (11.904761904761905em, -21.428571428571427em) {$\bullet$} ; 
\node[inner sep=5pt] (4) at (14.238095238095237em, -21.37202380952381em) {$\textcolor{red}{=}$} ; 
\path 
(0) to[fore, black,->, into, ] node[coordinate](5){} (1) 
(1) to[fore, black,->, into, ] node[coordinate](6){} (2) 
(3) to[fore, black,->, curve={ratio=-0.2}, ] node[coordinate](7){} (0) 
(3) to[fore, black,->, curve={ratio=-0.6}, ] node[coordinate](8){} (2) 
(0) to[fore, black,->, curve={ratio=-0.6}, ] node[coordinate](9){} (2) 
(3) to[fore, red,->, curve={ratio=0.6}, ] node[coordinate](10){} (1) 
(3) to[fore, black,->, curve={ratio=0.2}, ] node[coordinate](11){} (0) 
(1) to[fore, black,->, cell=0.2, curve={ratio=0.2}, ] node[coordinate](12){} (9) 
(0) to[fore, black,->, cell=0.2, curve={ratio=-0.4}, ] node[coordinate](13){} (8) 
(0) to[fore, black,->, cell=0.2, curve={ratio=0.1}, ] node[coordinate](14){} (8) 
(0) to[fore, red,->, cell=0.2, curve={ratio=-0.2}, ] node[coordinate](15){} (10) 
(1) to[fore, red,->, cell=0.2, curve={ratio=-0.2}, ] node[coordinate](16){} (8) 
(0) to[fore, red,->, cell=0.2, curve={ratio=0.4}, ] node[coordinate](17){} (10) 
(1) to[fore, red,->, cell=0.2, curve={ratio=0.2}, ] node[coordinate](18){} (8) 
; 
\path[->] 
(0) edge["${\scriptstyle }$", pos=0.5, fore, black,->, into, ] (1) 
(1) edge["${\scriptstyle }$", pos=0.5, fore, black,->, into, ] (2) 
(3) edge["${\scriptstyle }$", pos=0.5, fore, black,->, curve={ratio=-0.2}, ] (0) 
(3) edge["${\scriptstyle }$", pos=0.5, fore, black,->, curve={ratio=-0.6}, ] (2) 
(0) edge["${\scriptstyle }$", pos=0.5, fore, black,->, curve={ratio=-0.6}, ] (2) 
(3) edge["${\scriptstyle }$"', pos=0.5, fore, red,->, curve={ratio=0.6}, ] (1) 
(3) edge["${\scriptstyle }$", pos=0.5, fore, black,->, curve={ratio=0.2}, ] (0) 
(1) edge["${\scriptstyle }$", pos=0.5, fore, black,->, cell=0.2, curve={ratio=0.2}, ] (9) 
(0) edge["${\scriptstyle }$", pos=0.5, fore, black,->, cell=0.2, curve={ratio=-0.4}, ] (8) 
(0) edge["${\scriptstyle }$", pos=0.5, fore, black,->, cell=0.2, curve={ratio=0.1}, ] (8) 
(0) edge["${\scriptstyle }$", pos=0.5, fore, red,->, cell=0.2, curve={ratio=-0.2}, ] (10) 
(1) edge["${\scriptstyle }$", pos=0.5, fore, red,->, cell=0.2, curve={ratio=-0.2}, ] (8) 
(0) edge["${\scriptstyle }$", pos=0.5, fore, red,->, cell=0.2, curve={ratio=0.4}, ] (10) 
(1) edge["${\scriptstyle }$", pos=0.5, fore, red,->, cell=0.2, curve={ratio=0.2}, ] (8) 
; 
\end{tikzpicture}
    % END OF GENERATED LATEX
    \label{eq:compo-mono-simple-rule}
\end{align}
By removing distinguished elements, we recover the above rewriting rule \eqref{eq:compo-mono-simplest-rule}.

\subsubsection{Second-order rewriting rules}
The admissible rewriting rule \eqref{eq:compo-mono-simplest-rule} shows that the composition of two monomorphisms is monomorphic.
This makes following rewriting rule legitimate:
\[
% YADE DIAGRAM compo-mono-second-order.yade
% GENERATED LATEX
\begin{tikzpicture}[every node/.style={outer sep=0pt,anchor=base,text height=1.2ex, text depth=0.25ex}] 
\node[inner sep=5pt] (0) at (11.904761904761905em, -7.142857142857143em) {$\bullet$} ; 
\node[inner sep=5pt] (1) at (16.666666666666668em, -7.142857142857143em) {$\bullet$} ; 
\node[inner sep=5pt] (2) at (21.428571428571427em, -7.142857142857143em) {$\bullet$} ; 
\path 
(0) to[fore, black,->, into, ] node[coordinate](3){} (1) 
(1) to[fore, black,->, into, ] node[coordinate](4){} (2) 
(0) to[fore, black,->, curve={ratio=-0.5}, ] node[coordinate](5){} (2) 
(0) to[fore, red,->, curve={ratio=-0.5}, into, ] node[coordinate](6){} (2) 
(1) to[fore, black,->, cell=0.2, ] node[coordinate](7){} (5) 
; 
\path[->] 
(0) edge["${\scriptstyle }$", pos=0.5, fore, black,->, into, ] (1) 
(1) edge["${\scriptstyle }$", pos=0.5, fore, black,->, into, ] (2) 
(0) edge["${\scriptstyle }$", pos=0.5, fore, red,->, curve={ratio=-0.5}, into, ] (2) 
(1) edge["${\scriptstyle }$", pos=0.5, fore, black,->, cell=0.2, ] (5) 
(0) edge["${\scriptstyle }$", pos=0.5, fore, black,->, curve={ratio=-0.5}, ] (2) 
; 
\end{tikzpicture}
% END OF GENERATED LATEX
\]
Unfortunately, there is no way to deduce this rule using the chaining and pruning mechanisms.

The formula that we would like to account is of the following shape:
\[
% YADE DIAGRAM mono-so-formula.yade
% GENERATED LATEX
\begin{tikzpicture}[every node/.style={outer sep=0pt,anchor=base,text height=1.2ex, text depth=0.25ex}] 
\node[inner sep=5pt] (0) at (15.729166666666668em, -7.802734375em) {$\forall$} ; 
\node[inner sep=5pt] (1) at (18.229166666666668em, -7.8125em) {$\bullet$} ; 
\node[inner sep=5pt] (2) at (23.4375em, -7.8125em) {$\bullet$} ; 
\node[inner sep=5pt] (3) at (18.854166666666668em, -11.761067708333334em) {$(\textcolor{blue}{\forall }$} ; 
\node[inner sep=5pt] (4) at (28.645833333333336em, -13.020833333333334em) {$\bullet$} ; 
\node[inner sep=5pt] (5) at (33.85416666666667em, -13.020833333333334em) {$\bullet$} ; 
\node[inner sep=5pt] (6) at (23.4375em, -13.020833333333334em) {$\textcolor{blue}{\bullet}$} ; 
\node[inner sep=5pt] (7) at (37.60416666666667em, -11.969401041666668em) {$\textcolor{red}{\exists}$} ; 
\node[inner sep=5pt] (8) at (46.145833333333336em, -12.958984375em) {$\bullet$} ; 
\node[inner sep=5pt] (9) at (51.35416666666667em, -12.958984375em) {$\bullet$} ; 
\node[inner sep=5pt] (10) at (40.9375em, -12.958984375em) {$\textcolor{blue}{\bullet}$} ; 
\node[inner sep=5pt] (11) at (43.489583333333336em, -12.906901041666668em) {$\textcolor{red}{=}$} ; 
\node[inner sep=5pt] (12) at (53.489583333333336em, -11.813151041666668em) {$)$} ; 
\node[inner sep=5pt] (13) at (18.802083333333336em, -17.854817708333336em) {$\Rightarrow$} ; 
\node[inner sep=5pt] (14) at (22.34375em, -18.375651041666668em) {$\textcolor{red}{\exists}$} ; 
\node[inner sep=5pt] (15) at (28.645833333333336em, -18.229166666666668em) {$\bullet$} ; 
\node[inner sep=5pt] (16) at (33.85416666666667em, -18.229166666666668em) {$\bullet$} ; 
\path 
(1) to[fore, black,->, ] node[coordinate](17){} (2) 
(4) to[fore, black,->, ] node[coordinate](18){} (5) 
(6) to[fore, blue,->, curve={ratio=-0.2}, ] node[coordinate](19){} (4) 
(6) to[fore, blue,->, curve={ratio=0.1}, ] node[coordinate](20){} (4) 
(6) to[fore, blue,->, curve={ratio=-0.6}, ] node[coordinate](21){} (5) 
(8) to[fore, black,->, ] node[coordinate](22){} (9) 
(10) to[fore, blue,->, curve={ratio=-0.2}, ] node[coordinate](23){} (8) 
(10) to[fore, blue,->, curve={ratio=0.2}, ] node[coordinate](24){} (8) 
(10) to[fore, blue,->, curve={ratio=-0.6}, ] node[coordinate](25){} (9) 
(15) to[fore, black,->, ] node[coordinate](26){} (16) 
(15) to[fore, red,->, into, ] node[coordinate](27){} (16) 
(4) to[fore, blue,->, cell=0, curve={ratio=-0.2}, ] node[coordinate](28){} (21) 
(4) to[fore, blue,->, cell=0, curve={ratio=0.4}, ] node[coordinate](29){} (21) 
(8) to[fore, blue,->, cell=0, curve={ratio=-0.2}, ] node[coordinate](30){} (25) 
(8) to[fore, blue,->, cell=0, curve={ratio=0.4}, ] node[coordinate](31){} (25) 
; 
\path[->, transform shape, every edge quotes/.style={}, scale=0.7] 
(15) edge["${ }$" auto=left, pos=0.5, fore, red,->, into, ] (16) 
(1) edge["${ f}$" auto=left, pos=0.5, fore, black,->, ] (2) 
(4) edge["${ f}$" auto=left, pos=0.5, fore, black,->, ] (5) 
(6) edge["${ }$" auto=left, pos=0.5, fore, blue,->, curve={ratio=-0.2}, ] (4) 
(6) edge["${ }$" auto=right, pos=0.5, fore, blue,->, curve={ratio=0.1}, ] (4) 
(6) edge["${ }$" auto=left, pos=0.5, fore, blue,->, curve={ratio=-0.6}, ] (5) 
(8) edge["${ f}$" auto=left, pos=0.5, fore, black,->, ] (9) 
(10) edge["${ }$" auto=left, pos=0.5, fore, blue,->, curve={ratio=-0.2}, ] (8) 
(10) edge["${ }$" auto=right, pos=0.5, fore, blue,->, curve={ratio=0.2}, ] (8) 
(10) edge["${ }$" auto=left, pos=0.5, fore, blue,->, curve={ratio=-0.6}, ] (9) 
(15) edge["${ f}$" auto=left, pos=0.5, fore, black,->, ] (16) 
(4) edge["${ }$" auto=left, pos=0.5, fore, blue,->, cell=0, curve={ratio=-0.2}, ] (21) 
(4) edge["${ }$" auto=left, pos=0.5, fore, blue,->, cell=0, curve={ratio=0.4}, ] (21) 
(8) edge["${ }$" auto=left, pos=0.5, fore, blue,->, cell=0, curve={ratio=-0.2}, ] (25) 
(8) edge["${ }$" auto=left, pos=0.5, fore, blue,->, cell=0, curve={ratio=0.4}, ] (25) 
; 
\end{tikzpicture}
% END OF GENERATED LATEX
\] 
It does not fit the format $∀ .. ∃ .. P$ of our previous rewriting rules (see \Cref{rem:formula-fo}).

This motivates what we call \emph{second-order rewriting rules}, which handles formulae of the shape above, where 
the second line can occur many time. That is,we handle formulae of the shape $∀ (∀ ∃ .. )^* ∃ .. $.
    \subsubsection{Third-order order rules and beyond}
One may worry that second-order rules are not strong enough to capture all the possible rewriting rules that we want. What if we want additional nesting of quantifiers, like the third-order rewriting rule below?
\begin{equation}
    \label{eq:third-order}
∀ x. \left(∀y. (∀z ∃ w.P) ⇒ ∃..Q  \right) 
⇒ ∃.. R 
\end{equation}
In fact, we can "emulate" them with second-order rewriting rules with a trick: we can introduce a new sort, and use it to represent the inner quantifiers\footnote{Intriguingly enough, github copilot correctly guessed the end of this sentence, after I wrote "emulate them".}. 

Let us illustrate this trick with \Cref{eq:third-order}.
We introduce a new sort $S$ depending on $x,y$ such that
\begin{equation}
    \label{eq:sort-equivalence}
∀ xy\left(∃ s:S(x,y)\   ⇔ \ 
∀ z ∃ w P\right) 
\end{equation}
so that \Cref{eq:third-order} is equivalent to the following second-order rule:
\[
    ∀ x. \left(∀y(s:S(x,y)) ∃..Q  \right) ⇒ ∃.. R
\]

Now, let us explain how we can enforce the equivalence of
\Cref{eq:sort-equivalence}.
For the left-to-right implication, a first-order rewriting rule is enough:
\[
    ∀ xy (s:S(x,y))z   \ 
    ∃ w. P
\]
For the reverse implication, we can write a second order rewriting rule:
\[
    ∀xy(∀ z ∃ w. P)\   ⇒ ∃s:S(x,y)  \
\]

\section{Internal representation}
A diagram is stored as a graph: 
each element in the graph is a node, and each dependency is an edge.
For example, the following diagram below left is represented as the graph below right:
\[
% YADE DIAGRAM representation-graph.yade
% GENERATED LATEX
\begin{tikzpicture}[every node/.style={outer sep=0pt,anchor=base,text height=1.2ex, text depth=0.25ex}] 
\node[inner sep=5pt] (0) at (11.904761904761905em, -11.904761904761905em) {$\bullet$} ; 
\node[inner sep=5pt] (1) at (16.666666666666668em, -7.142857142857143em) {$\bullet$} ; 
\node[inner sep=5pt] (2) at (21.428571428571427em, -11.904761904761905em) {$\bullet$} ; 
\node[inner sep=5pt] (3) at (23.904761904761905em, -5.038690476190476em) {$ $} ; 
\node[inner sep=5pt] (4) at (23.904761904761905em, -14.133928571428571em) {$ $} ; 
\node[inner sep=5pt] (5) at (28.61904761904762em, -12.467261904761905em) {$ob$} ; 
\node[inner sep=5pt] (6) at (42.38095238095238em, -12.37202380952381em) {$ob$} ; 
\node[inner sep=5pt] (7) at (35.666666666666664em, -12.467261904761905em) {$mor$} ; 
\node[inner sep=5pt] (8) at (35.333333333333336em, -4.705357142857143em) {$ob$} ; 
\node[inner sep=5pt] (9) at (39.76190476190476em, -8.705357142857142em) {$mor$} ; 
\node[inner sep=5pt] (10) at (31.19047619047619em, -8.75297619047619em) {$mor$} ; 
\node[inner sep=5pt] (11) at (26.238095238095237em, -5.705357142857143em) {$mono$} ; 
\node[inner sep=5pt] (12) at (35.714285714285715em, -9.419642857142858em) {$triangle$} ; 
\path 
(0) to[fore, black,->, into, ] node[coordinate](13){} (1) 
(1) to[fore, black,->, ] node[coordinate](14){} (2) 
(0) to[fore, black,->, ] node[coordinate](15){} (2) 
(3) to[fore, black,-,] node[coordinate](16){} (4) 
(7) to[fore, black,->, ] node[coordinate](17){} (5) 
(7) to[fore, black,->, ] node[coordinate](18){} (6) 
(9) to[fore, black,->, ] node[coordinate](19){} (8) 
(9) to[fore, black,->, ] node[coordinate](20){} (6) 
(10) to[fore, black,->, ] node[coordinate](21){} (8) 
(10) to[fore, black,->, ] node[coordinate](22){} (5) 
(11) to[fore, black,->, ] node[coordinate](23){} (10) 
(12) to[fore, black,->, ] node[coordinate](24){} (8) 
(12) to[fore, black,->, ] node[coordinate](25){} (10) 
(12) to[fore, black,->, ] node[coordinate](26){} (5) 
(12) to[fore, black,->, ] node[coordinate](27){} (7) 
(12) to[fore, black,->, ] node[coordinate](28){} (6) 
(12) to[fore, black,->, ] node[coordinate](29){} (9) 
(11) to[fore, black,->, curve={ratio=-0.1}, ] node[coordinate](30){} (8) 
(11) to[fore, black,->, curve={ratio=0.2}, ] node[coordinate](31){} (5) 
(1) to[fore, black,->, cell=0.2, ] node[coordinate](32){} (15) 
; 
\path[->] 
(0) edge["${\scriptstyle }$", pos=0.5, fore, black,->, into, ] (1) 
(1) edge["${\scriptstyle }$", pos=0.5, fore, black,->, ] (2) 
(0) edge["${\scriptstyle }$", pos=0.5, fore, black,->, ] (2) 
(3) edge["${\scriptstyle }$", pos=0.5, fore, black,-,] (4) 
(7) edge["${\scriptstyle source}$", pos=0.5, fore, black,->, ] (5) 
(7) edge["${\scriptstyle target}$"', pos=0.5, fore, black,->, ] (6) 
(9) edge["${\scriptstyle source}$"', pos=0.5, fore, black,->, ] (8) 
(9) edge["${\scriptstyle target}$", pos=0.5, fore, black,->, ] (6) 
(10) edge["${\scriptstyle target}$", pos=0.5, fore, black,->, ] (8) 
(10) edge["${\scriptstyle source}$"', pos=0.5, fore, black,->, ] (5) 
(11) edge["${\scriptstyle }$", pos=0.5, fore, black,->, ] (10) 
(12) edge["${\scriptstyle }$", pos=0.5, fore, black,->, ] (8) 
(12) edge["${\scriptstyle }$", pos=0.5, fore, black,->, ] (10) 
(12) edge["${\scriptstyle }$", pos=0.5, fore, black,->, ] (5) 
(12) edge["${\scriptstyle }$", pos=0.5, fore, black,->, ] (7) 
(12) edge["${\scriptstyle }$", pos=0.5, fore, black,->, ] (6) 
(12) edge["${\scriptstyle }$", pos=0.5, fore, black,->, ] (9) 
(11) edge["${\scriptstyle }$", pos=0.5, fore, black,->, curve={ratio=-0.1}, ] (8) 
(11) edge["${\scriptstyle }$", pos=0.5, fore, black,->, curve={ratio=0.2}, ] (5) 
(1) edge["${\scriptstyle }$", pos=0.5, fore, black,->, cell=0.2, ] (15) 
; 
\end{tikzpicture}
% END OF GENERATED LATEX
\]
Looking for a match can be done at the level of this graph representation.
\section{Semantics}
A sort specification corresponds to a finite direct category. 
A diagram with distinguished elements corresponds to a 
natural transformation between two functors from this category to the category of sets.
\appendix
\section{Attempts to simplify the proof that monomorphisms compose}
The last attempt is pretty convincing; it we require tactics.
\label{app:simplified-proof}
\paragraph{Simplifying the above example}
First we can construct the following associativity rule.
\begin{align}
    % YADE DIAGRAM associativity-better.yade
    % GENERATED LATEX
    \begin{tikzpicture}[every node/.style={outer sep=0pt,anchor=base,text height=1.2ex, text depth=0.25ex}] 
\node[inner sep=5pt] (0) at (16.666666666666668em, -11.904761904761905em) {$\bullet$} ; 
\node[inner sep=5pt] (1) at (21.428571428571427em, -11.904761904761905em) {$\bullet$} ; 
\node[inner sep=5pt] (2) at (26.19047619047619em, -11.904761904761905em) {$\bullet$} ; 
\node[inner sep=5pt] (3) at (30.952380952380953em, -11.904761904761905em) {$\bullet$} ; 
\path 
(0) to[fore, black,->, ] node[coordinate](4){} (1) 
(1) to[fore, black,->, ] node[coordinate](5){} (2) 
(2) to[fore, black,->, ] node[coordinate](6){} (3) 
(1) to[fore, black,->, curve={ratio=-0.4}, ] node[coordinate](7){} (3) 
(0) to[fore, black,->, curve={ratio=-0.7999999999999999}, ] node[coordinate](8){} (3) 
(0) to[fore, red,->, curve={ratio=0.6}, ] node[coordinate](9){} (2) 
(2) to[fore, black,->, cell=0.2, curve={ratio=0.5}, ] node[coordinate](10){} (7) 
(1) to[fore, black,->, cell=0.2, curve={ratio=-0.1}, ] node[coordinate](11){} (8) 
(1) to[fore, red,->, cell=0.2, ] node[coordinate](12){} (9) 
(2) to[fore, red,->, cell=0.2, curve={ratio=0.1}, ] node[coordinate](13){} (8) 
; 
\path[->] 
(0) edge["${\scriptstyle }$", pos=0.5, fore, black,->, ] (1) 
(1) edge["${\scriptstyle }$", pos=0.5, fore, black,->, ] (2) 
(2) edge["${\scriptstyle }$", pos=0.5, fore, black,->, ] (3) 
(1) edge["${\scriptstyle }$", pos=0.5, fore, black,->, curve={ratio=-0.4}, ] (3) 
(0) edge["${\scriptstyle }$", pos=0.5, fore, black,->, curve={ratio=-0.7999999999999999}, ] (3) 
(0) edge["${\scriptstyle }$", pos=0.5, fore, red,->, curve={ratio=0.6}, ] (2) 
(2) edge["${\scriptstyle }$", pos=0.5, fore, black,->, cell=0.2, curve={ratio=0.5}, ] (7) 
(1) edge["${\scriptstyle }$", pos=0.5, fore, black,->, cell=0.2, curve={ratio=-0.1}, ] (8) 
(1) edge["${\scriptstyle }$", pos=0.5, fore, red,->, cell=0.2, ] (9) 
(2) edge["${\scriptstyle }$", pos=0.5, fore, red,->, cell=0.2, curve={ratio=0.1}, ] (8) 
; 
\end{tikzpicture}
    % END OF GENERATED LATEX
    \tag{Comp-Assoc}
    \label{eq:comp-assoc}
\end{align}
Let us derive it more carefully.
\begin{align*}
    % YADE DIAGRAM compo-mono-ws-2-proof.yade
    % GENERATED LATEX
    \begin{tikzpicture}[every node/.style={outer sep=0pt,anchor=base,text height=1.2ex, text depth=0.25ex}] 
\node[inner sep=5pt] (0) at (2.380952380952381em, -11.904761904761905em) {$\bullet$} ; 
\node[inner sep=5pt] (1) at (7.142857142857143em, -11.904761904761905em) {$\bullet$} ; 
\node[inner sep=5pt] (2) at (11.904761904761905em, -11.904761904761905em) {$\bullet$} ; 
\node[inner sep=5pt] (3) at (16.666666666666668em, -11.904761904761905em) {$\bullet$} ; 
\node[inner sep=5pt] (4) at (19.095238095238095em, -4.5625em) {$ $} ; 
\node[inner sep=5pt] (5) at (19.047619047619047em, -14.800595238095237em) {$\eqref{eq:rew-comp}$} ; 
\node[inner sep=5pt] (6) at (21.428571428571427em, -11.904761904761905em) {$\bullet$} ; 
\node[inner sep=5pt] (7) at (26.19047619047619em, -11.904761904761905em) {$\bullet$} ; 
\node[inner sep=5pt] (8) at (30.952380952380953em, -11.904761904761905em) {$\bullet$} ; 
\node[inner sep=5pt] (9) at (35.714285714285715em, -11.904761904761905em) {$\bullet$} ; 
\node[inner sep=5pt] (10) at (2.380952380952381em, -26.19047619047619em) {$\bullet$} ; 
\node[inner sep=5pt] (11) at (7.142857142857143em, -26.19047619047619em) {$\bullet$} ; 
\node[inner sep=5pt] (12) at (11.904761904761905em, -26.19047619047619em) {$\bullet$} ; 
\node[inner sep=5pt] (13) at (16.666666666666668em, -26.19047619047619em) {$\bullet$} ; 
\node[inner sep=5pt] (14) at (38.023809523809526em, -4.300595238095238em) {$ $} ; 
\node[inner sep=5pt] (15) at (37.976190476190474em, -14.538690476190476em) {$\eqref{eq:rew-comp}$} ; 
\node[inner sep=5pt] (16) at (19.11904761904762em, -20.87202380952381em) {$ $} ; 
\node[inner sep=5pt] (17) at (19.071428571428573em, -31.110119047619047em) {$\eqref{eq:rew-assoc}$} ; 
\node[inner sep=5pt] (18) at (21.428571428571427em, -26.19047619047619em) {$\bullet$} ; 
\node[inner sep=5pt] (19) at (26.19047619047619em, -26.19047619047619em) {$\bullet$} ; 
\node[inner sep=5pt] (20) at (30.952380952380953em, -26.19047619047619em) {$\bullet$} ; 
\node[inner sep=5pt] (21) at (35.714285714285715em, -26.19047619047619em) {$\bullet$} ; 
\path 
(0) to[fore, blue,->, ] node[coordinate](22){} (1) 
(1) to[fore, blue,->, ] node[coordinate](23){} (2) 
(2) to[fore, black,->, ] node[coordinate](24){} (3) 
(1) to[fore, black,->, curve={ratio=-0.5}, ] node[coordinate](25){} (3) 
(0) to[fore, black,->, curve={ratio=-0.7999999999999999}, ] node[coordinate](26){} (3) 
(4) to[fore, black,-,] node[coordinate](27){} (5) 
(6) to[fore, black,->, ] node[coordinate](28){} (7) 
(7) to[fore, black,->, ] node[coordinate](29){} (8) 
(8) to[fore, blue,->, ] node[coordinate](30){} (9) 
(7) to[fore, black,->, curve={ratio=-0.5}, ] node[coordinate](31){} (9) 
(6) to[fore, black,->, curve={ratio=-0.7999999999999999}, ] node[coordinate](32){} (9) 
(6) to[fore, orange,->, curve={ratio=0.5}, ] node[coordinate](33){} (8) 
(10) to[fore, blue,->, ] node[coordinate](34){} (11) 
(12) to[fore, blue,->, ] node[coordinate](35){} (13) 
(11) to[fore, blue,->, curve={ratio=-0.5}, ] node[coordinate](36){} (13) 
(10) to[fore, blue,->, curve={ratio=-0.7999999999999999}, ] node[coordinate](37){} (13) 
(10) to[fore, blue,->, curve={ratio=0.5}, ] node[coordinate](38){} (12) 
(14) to[fore, black,-,] node[coordinate](39){} (15) 
(11) to[fore, blue,->, ] node[coordinate](40){} (12) 
(10) to[fore, orange,->, curve={ratio=0.7999999999999999}, ] node[coordinate](41){} (13) 
(16) to[fore, black,-,] node[coordinate](42){} (17) 
(18) to[fore, black,->, ] node[coordinate](43){} (19) 
(19) to[fore, black,->, ] node[coordinate](44){} (20) 
(20) to[fore, black,->, ] node[coordinate](45){} (21) 
(19) to[fore, black,->, curve={ratio=-0.5}, ] node[coordinate](46){} (21) 
(18) to[fore, black,->, curve={ratio=-0.7999999999999999}, ] node[coordinate](47){} (21) 
(18) to[fore, black,->, curve={ratio=0.6}, ] node[coordinate](48){} (20) 
(2) to[fore, black,->, cell=0.2, ] node[coordinate](49){} (25) 
(1) to[fore, black,->, cell=0.2, curve={ratio=-0.1}, ] node[coordinate](50){} (26) 
(8) to[fore, black,->, cell=0.2, ] node[coordinate](51){} (31) 
(7) to[fore, black,->, cell=0.2, curve={ratio=-0.1}, ] node[coordinate](52){} (32) 
(7) to[fore, red,->, cell=0.2, ] node[coordinate](53){} (33) 
(11) to[fore, blue,->, cell=0.2, curve={ratio=-0.1}, ] node[coordinate](54){} (37) 
(11) to[fore, blue,->, cell=0.2, ] node[coordinate](55){} (38) 
(12) to[fore, orange,->, cell=0.2, ] node[coordinate](56){} (41) 
(12) to[fore, blue,->, cell=0.2, ] node[coordinate](57){} (36) 
(20) to[fore, black,->, cell=0.2, curve={ratio=0.5}, ] node[coordinate](58){} (46) 
(19) to[fore, black,->, cell=0.2, curve={ratio=-0.1}, ] node[coordinate](59){} (47) 
(19) to[fore, black,->, cell=0.2, ] node[coordinate](60){} (48) 
(20) to[fore, black,->, cell=0.2, curve={ratio=0.1}, ] node[coordinate](61){} (47) 
; 
\path[->] 
(0) edge["${\scriptstyle }$", pos=0.5, fore, blue,->, ] (1) 
(1) edge["${\scriptstyle }$", pos=0.5, fore, blue,->, ] (2) 
(2) edge["${\scriptstyle }$", pos=0.5, fore, black,->, ] (3) 
(1) edge["${\scriptstyle }$", pos=0.5, fore, black,->, curve={ratio=-0.5}, ] (3) 
(0) edge["${\scriptstyle }$", pos=0.5, fore, black,->, curve={ratio=-0.7999999999999999}, ] (3) 
(4) edge["${\scriptstyle }$", pos=0.5, fore, black,-,] (5) 
(6) edge["${\scriptstyle }$", pos=0.5, fore, black,->, ] (7) 
(7) edge["${\scriptstyle }$", pos=0.5, fore, black,->, ] (8) 
(8) edge["${\scriptstyle }$", pos=0.5, fore, blue,->, ] (9) 
(7) edge["${\scriptstyle }$", pos=0.5, fore, black,->, curve={ratio=-0.5}, ] (9) 
(6) edge["${\scriptstyle }$", pos=0.5, fore, black,->, curve={ratio=-0.7999999999999999}, ] (9) 
(6) edge["${\scriptstyle }$", pos=0.5, fore, orange,->, curve={ratio=0.5}, ] (8) 
(10) edge["${\scriptstyle }$", pos=0.5, fore, blue,->, ] (11) 
(12) edge["${\scriptstyle }$", pos=0.5, fore, blue,->, ] (13) 
(11) edge["${\scriptstyle }$", pos=0.5, fore, blue,->, curve={ratio=-0.5}, ] (13) 
(10) edge["${\scriptstyle }$", pos=0.5, fore, blue,->, curve={ratio=-0.7999999999999999}, ] (13) 
(10) edge["${\scriptstyle }$", pos=0.5, fore, blue,->, curve={ratio=0.5}, ] (12) 
(14) edge["${\scriptstyle }$", pos=0.5, fore, black,-,] (15) 
(11) edge["${\scriptstyle }$", pos=0.5, fore, blue,->, ] (12) 
(10) edge["${\scriptstyle }$", pos=0.5, fore, orange,->, curve={ratio=0.7999999999999999}, ] (13) 
(16) edge["${\scriptstyle }$", pos=0.5, fore, black,-,] (17) 
(18) edge["${\scriptstyle }$", pos=0.5, fore, black,->, ] (19) 
(19) edge["${\scriptstyle }$", pos=0.5, fore, black,->, ] (20) 
(20) edge["${\scriptstyle }$", pos=0.5, fore, black,->, ] (21) 
(19) edge["${\scriptstyle }$", pos=0.5, fore, black,->, curve={ratio=-0.5}, ] (21) 
(18) edge["${\scriptstyle }$", pos=0.5, fore, black,->, curve={ratio=-0.7999999999999999}, ] (21) 
(18) edge["${\scriptstyle }$", pos=0.5, fore, black,->, curve={ratio=0.6}, ] (20) 
(2) edge["${\scriptstyle }$", pos=0.5, fore, black,->, cell=0.2, ] (25) 
(1) edge["${\scriptstyle }$", pos=0.5, fore, black,->, cell=0.2, curve={ratio=-0.1}, ] (26) 
(8) edge["${\scriptstyle }$", pos=0.5, fore, black,->, cell=0.2, ] (31) 
(7) edge["${\scriptstyle }$", pos=0.5, fore, black,->, cell=0.2, curve={ratio=-0.1}, ] (32) 
(7) edge["${\scriptstyle }$", pos=0.5, fore, red,->, cell=0.2, ] (33) 
(11) edge["${\scriptstyle }$", pos=0.5, fore, blue,->, cell=0.2, curve={ratio=-0.1}, ] (37) 
(11) edge["${\scriptstyle }$", pos=0.5, fore, blue,->, cell=0.2, ] (38) 
(12) edge["${\scriptstyle }$", pos=0.5, fore, orange,->, cell=0.2, ] (41) 
(12) edge["${\scriptstyle }$", pos=0.5, fore, blue,->, cell=0.2, ] (36) 
(20) edge["${\scriptstyle }$", pos=0.5, fore, black,->, cell=0.2, curve={ratio=0.5}, ] (46) 
(19) edge["${\scriptstyle }$", pos=0.5, fore, black,->, cell=0.2, curve={ratio=-0.1}, ] (47) 
(19) edge["${\scriptstyle }$", pos=0.5, fore, black,->, cell=0.2, ] (48) 
(20) edge["${\scriptstyle }$", pos=0.5, fore, black,->, cell=0.2, curve={ratio=0.1}, ] (47) 
; 
\end{tikzpicture}
    % END OF GENERATED LATEX
\end{align*}

We present a second version of the long proof of \eqref{eq:compo-mono-simple-rule}.
\pagebreak
\[
% YADE DIAGRAM compo-mono-ws-2.yade
% GENERATED LATEX
\begin{tikzpicture}[every node/.style={outer sep=0pt,anchor=base,text height=1.2ex, text depth=0.25ex}] 
\node[inner sep=5pt] (0) at (9.571428571428571em, -1.0863095238095237em) {$ $} ; 
\node[inner sep=5pt] (1) at (11.904761904761905em, -7.142857142857143em) {$\textcolor{black}{\bullet}$} ; 
\node[inner sep=5pt] (2) at (16.666666666666668em, -7.142857142857143em) {$\textcolor{black}{\bullet}$} ; 
\node[inner sep=5pt] (3) at (21.428571428571427em, -7.142857142857143em) {$\textcolor{black}{\bullet}$} ; 
\node[inner sep=5pt] (4) at (23.738095238095237em, -2.3720238095238093em) {$ $} ; 
\node[inner sep=5pt] (5) at (23.785714285714285em, -10.276785714285714em) {$\eqref{eq:comp-assoc}$} ; 
\node[inner sep=5pt] (6) at (7.142857142857143em, -7.142857142857143em) {$\bullet$} ; 
\node[inner sep=5pt] (7) at (30.952380952380953em, -7.142857142857143em) {$\textcolor{black}{\bullet}$} ; 
\node[inner sep=5pt] (8) at (35.714285714285715em, -7.142857142857143em) {$\textcolor{black}{\bullet}$} ; 
\node[inner sep=5pt] (9) at (40.476190476190474em, -7.142857142857143em) {$\textcolor{black}{\bullet}$} ; 
\node[inner sep=5pt] (10) at (26.19047619047619em, -7.142857142857143em) {$\bullet$} ; 
\node[inner sep=5pt] (11) at (42.833333333333336em, -2.6577380952380953em) {$ $} ; 
\node[inner sep=5pt] (12) at (42.88095238095238em, -9.75297619047619em) {$\eqref{eq:comp-assoc}$} ; 
\node[inner sep=5pt] (13) at (23.738095238095237em, -12.276785714285714em) {$ $} ; 
\node[inner sep=5pt] (14) at (42.92857142857143em, -11.657738095238095em) {$ $} ; 
\node[inner sep=5pt] (15) at (11.904761904761905em, -21.428571428571427em) {$\textcolor{black}{\bullet}$} ; 
\node[inner sep=5pt] (16) at (16.666666666666668em, -21.428571428571427em) {$\textcolor{black}{\bullet}$} ; 
\node[inner sep=5pt] (17) at (21.428571428571427em, -21.428571428571427em) {$\textcolor{black}{\bullet}$} ; 
\node[inner sep=5pt] (18) at (7.142857142857143em, -21.428571428571427em) {$\bullet$} ; 
\node[inner sep=5pt] (19) at (23.88095238095238em, -17.276785714285715em) {$ $} ; 
\node[inner sep=5pt] (20) at (23.928571428571427em, -25.18154761904762em) {$\eqref{eq:rew-mono-fo}$} ; 
\node[inner sep=5pt] (21) at (30.952380952380953em, -21.428571428571427em) {$\textcolor{black}{\bullet}$} ; 
\node[inner sep=5pt] (22) at (35.714285714285715em, -21.428571428571427em) {$\textcolor{black}{\bullet}$} ; 
\node[inner sep=5pt] (23) at (40.476190476190474em, -21.428571428571427em) {$\textcolor{black}{\bullet}$} ; 
\node[inner sep=5pt] (24) at (26.19047619047619em, -21.428571428571427em) {$\bullet$} ; 
\node[inner sep=5pt] (25) at (42.88095238095238em, -16.895833333333332em) {$ $} ; 
\node[inner sep=5pt] (26) at (42.92857142857143em, -24.800595238095237em) {$\eqref{eq:rew-mono-fo}$} ; 
\node[inner sep=5pt] (27) at (11.904761904761905em, -35.714285714285715em) {$\textcolor{black}{\bullet}$} ; 
\node[inner sep=5pt] (28) at (16.666666666666668em, -35.714285714285715em) {$\textcolor{black}{\bullet}$} ; 
\node[inner sep=5pt] (29) at (21.428571428571427em, -35.714285714285715em) {$\textcolor{black}{\bullet}$} ; 
\node[inner sep=5pt] (30) at (7.142857142857143em, -35.714285714285715em) {$\bullet$} ; 
\path 
(1) to[fore, blue,->, into, ] node[coordinate](31){} (2) 
(2) to[fore, blue,->, into, ] node[coordinate](32){} (3) 
(4) to[fore, black,-,] node[coordinate](33){} (5) 
(6) to[fore, blue,->, curve={ratio=0.2}, ] node[coordinate](34){} (1) 
(6) to[fore, black,->, curve={ratio=-0.30000000000000004}, ] node[coordinate](35){} (1) 
(6) to[fore, blue,->, curve={ratio=-0.6}, ] node[coordinate](36){} (3) 
(1) to[fore, blue,->, curve={ratio=-0.5}, ] node[coordinate](37){} (3) 
(7) to[fore, blue,->, into, ] node[coordinate](38){} (8) 
(8) to[fore, blue,->, into, ] node[coordinate](39){} (9) 
(10) to[fore, black,->, curve={ratio=0.2}, ] node[coordinate](40){} (7) 
(10) to[fore, blue,->, curve={ratio=-0.30000000000000004}, ] node[coordinate](41){} (7) 
(10) to[fore, blue,->, curve={ratio=-0.6}, ] node[coordinate](42){} (9) 
(7) to[fore, black,->, curve={ratio=-0.5}, ] node[coordinate](43){} (9) 
(10) to[fore, red,->, curve={ratio=0.6}, ] node[coordinate](44){} (8) 
(11) to[fore, black,-,] node[coordinate](45){} (12) 
(15) to[fore, black,->, into, ] node[coordinate](46){} (16) 
(16) to[fore, blue,->, into, ] node[coordinate](47){} (17) 
(18) to[fore, black,->, curve={ratio=0.2}, ] node[coordinate](48){} (15) 
(18) to[fore, black,->, curve={ratio=-0.30000000000000004}, ] node[coordinate](49){} (15) 
(18) to[fore, blue,->, curve={ratio=-0.6}, ] node[coordinate](50){} (17) 
(15) to[fore, black,->, curve={ratio=-0.6}, ] node[coordinate](51){} (17) 
(18) to[fore, blue,->, curve={ratio=0.6}, ] node[coordinate](52){} (16) 
(18) to[fore, orange,->, curve={ratio=0.7999999999999999}, ] node[coordinate](53){} (16) 
(19) to[fore, black,-,] node[coordinate](54){} (20) 
(21) to[fore, blue,->, into, ] node[coordinate](55){} (22) 
(22) to[fore, black,->, into, ] node[coordinate](56){} (23) 
(24) to[fore, blue,->, curve={ratio=0.2}, ] node[coordinate](57){} (21) 
(24) to[fore, blue,->, curve={ratio=-0.30000000000000004}, ] node[coordinate](58){} (21) 
(24) to[fore, black,->, curve={ratio=-0.6}, ] node[coordinate](59){} (23) 
(21) to[fore, black,->, curve={ratio=-0.6}, ] node[coordinate](60){} (23) 
(24) to[fore, blue,->, curve={ratio=0.6}, ] node[coordinate](61){} (22) 
(25) to[fore, black,-,] node[coordinate](62){} (26) 
(27) to[fore, black,->, into, ] node[coordinate](63){} (28) 
(28) to[fore, black,->, into, ] node[coordinate](64){} (29) 
(30) to[fore, black,->, ] node[coordinate](65){} (27) 
(30) to[fore, black,->, curve={ratio=-0.6}, ] node[coordinate](66){} (29) 
(27) to[fore, black,->, curve={ratio=-0.6}, ] node[coordinate](67){} (29) 
(30) to[fore, black,->, curve={ratio=0.6}, ] node[coordinate](68){} (28) 
(2) to[fore, blue,->, cell=0.2, ] node[coordinate](69){} (37) 
(1) to[fore, black,->, cell=0.2, curve={ratio=-0.4}, ] node[coordinate](70){} (36) 
(1) to[fore, blue,->, cell=0.2, curve={ratio=0.1}, ] node[coordinate](71){} (36) 
(8) to[fore, black,->, cell=0.2, ] node[coordinate](72){} (43) 
(7) to[fore, blue,->, cell=0.2, curve={ratio=-0.4}, ] node[coordinate](73){} (42) 
(7) to[fore, black,->, cell=0.2, curve={ratio=0.1}, ] node[coordinate](74){} (42) 
(7) to[fore, red,->, cell=0.2, ] node[coordinate](75){} (44) 
(16) to[fore, black,->, cell=0.2, curve={ratio=0.2}, ] node[coordinate](76){} (51) 
(15) to[fore, black,->, cell=0.2, curve={ratio=-0.4}, ] node[coordinate](77){} (50) 
(15) to[fore, black,->, cell=0.2, curve={ratio=0.1}, ] node[coordinate](78){} (50) 
(15) to[fore, black,->, cell=0.2, curve={ratio=-0.2}, ] node[coordinate](79){} (52) 
(15) to[fore, red,->, cell=0.2, curve={ratio=0.4}, ] node[coordinate](80){} (53) 
(16) to[fore, blue,->, cell=0.2, curve={ratio=-0.30000000000000004}, ] node[coordinate](81){} (50) 
(16) to[fore, orange,->, cell=0.2, curve={ratio=0.20000000000000004}, ] node[coordinate](82){} (50) 
(22) to[fore, black,->, cell=0.2, curve={ratio=0.2}, ] node[coordinate](83){} (60) 
(21) to[fore, black,->, cell=0.2, curve={ratio=-0.4}, ] node[coordinate](84){} (59) 
(21) to[fore, black,->, cell=0.2, curve={ratio=0.1}, ] node[coordinate](85){} (59) 
(21) to[fore, blue,->, cell=0.2, curve={ratio=-0.2}, ] node[coordinate](86){} (61) 
(22) to[fore, black,->, cell=0.2, curve={ratio=-0.2}, ] node[coordinate](87){} (59) 
(21) to[fore, blue,->, cell=0.2, curve={ratio=0.4}, ] node[coordinate](88){} (61) 
(22) to[fore, black,->, cell=0.2, curve={ratio=0.2}, ] node[coordinate](89){} (59) 
(28) to[fore, black,->, cell=0.2, curve={ratio=0.2}, ] node[coordinate](90){} (67) 
(27) to[fore, black,->, cell=0.2, curve={ratio=-0.4}, ] node[coordinate](91){} (66) 
(27) to[fore, black,->, cell=0.2, curve={ratio=0.1}, ] node[coordinate](92){} (66) 
(27) to[fore, black,->, cell=0.2, curve={ratio=-0.2}, ] node[coordinate](93){} (68) 
(28) to[fore, black,->, cell=0.2, curve={ratio=-0.2}, ] node[coordinate](94){} (66) 
(27) to[fore, black,->, cell=0.2, curve={ratio=0.4}, ] node[coordinate](95){} (68) 
(28) to[fore, black,->, cell=0.2, curve={ratio=0.2}, ] node[coordinate](96){} (66) 
(8) to[fore, blue,->, cell=0.2, ] node[coordinate](97){} (43) 
(8) to[fore, red,->, cell=0.2, curve={ratio=-0.2}, ] node[coordinate](98){} (42) 
; 
\path[->] 
(1) edge["${\scriptstyle }$", pos=0.5, fore, blue,->, into, ] (2) 
(2) edge["${\scriptstyle }$", pos=0.5, fore, blue,->, into, ] (3) 
(4) edge["${\scriptstyle }$", pos=0.5, fore, black,-,] (5) 
(6) edge["${\scriptstyle }$", pos=0.5, fore, blue,->, curve={ratio=0.2}, ] (1) 
(6) edge["${\scriptstyle }$", pos=0.5, fore, black,->, curve={ratio=-0.30000000000000004}, ] (1) 
(6) edge["${\scriptstyle }$", pos=0.5, fore, blue,->, curve={ratio=-0.6}, ] (3) 
(1) edge["${\scriptstyle }$", pos=0.5, fore, blue,->, curve={ratio=-0.5}, ] (3) 
(7) edge["${\scriptstyle }$", pos=0.5, fore, blue,->, into, ] (8) 
(8) edge["${\scriptstyle }$", pos=0.5, fore, blue,->, into, ] (9) 
(10) edge["${\scriptstyle }$", pos=0.5, fore, black,->, curve={ratio=0.2}, ] (7) 
(10) edge["${\scriptstyle }$", pos=0.5, fore, blue,->, curve={ratio=-0.30000000000000004}, ] (7) 
(10) edge["${\scriptstyle }$", pos=0.5, fore, blue,->, curve={ratio=-0.6}, ] (9) 
(7) edge["${\scriptstyle }$", pos=0.5, fore, black,->, curve={ratio=-0.5}, ] (9) 
(10) edge["${\scriptstyle }$"', pos=0.5, fore, red,->, curve={ratio=0.6}, ] (8) 
(11) edge["${\scriptstyle }$", pos=0.5, fore, black,-,] (12) 
(15) edge["${\scriptstyle }$", pos=0.5, fore, black,->, into, ] (16) 
(16) edge["${\scriptstyle }$", pos=0.5, fore, blue,->, into, ] (17) 
(18) edge["${\scriptstyle }$", pos=0.5, fore, black,->, curve={ratio=0.2}, ] (15) 
(18) edge["${\scriptstyle }$", pos=0.5, fore, black,->, curve={ratio=-0.30000000000000004}, ] (15) 
(18) edge["${\scriptstyle }$", pos=0.5, fore, blue,->, curve={ratio=-0.6}, ] (17) 
(15) edge["${\scriptstyle }$", pos=0.5, fore, black,->, curve={ratio=-0.6}, ] (17) 
(18) edge["${\scriptstyle }$"', pos=0.5, fore, blue,->, curve={ratio=0.6}, ] (16) 
(18) edge["${\scriptstyle }$", pos=0.5, fore, orange,->, curve={ratio=0.7999999999999999}, ] (16) 
(19) edge["${\scriptstyle }$", pos=0.5, fore, black,-,] (20) 
(21) edge["${\scriptstyle }$", pos=0.5, fore, blue,->, into, ] (22) 
(22) edge["${\scriptstyle }$", pos=0.5, fore, black,->, into, ] (23) 
(24) edge["${\scriptstyle }$", pos=0.5, fore, blue,->, curve={ratio=0.2}, ] (21) 
(24) edge["${\scriptstyle }$", pos=0.5, fore, blue,->, curve={ratio=-0.30000000000000004}, ] (21) 
(24) edge["${\scriptstyle }$", pos=0.5, fore, black,->, curve={ratio=-0.6}, ] (23) 
(21) edge["${\scriptstyle }$", pos=0.5, fore, black,->, curve={ratio=-0.6}, ] (23) 
(24) edge["${\scriptstyle }$"', pos=0.5, fore, blue,->, curve={ratio=0.6}, ] (22) 
(25) edge["${\scriptstyle }$", pos=0.5, fore, black,-,] (26) 
(27) edge["${\scriptstyle }$", pos=0.5, fore, black,->, into, ] (28) 
(28) edge["${\scriptstyle }$", pos=0.5, fore, black,->, into, ] (29) 
(30) edge["${\scriptstyle }$", pos=0.5, fore, black,->, ] (27) 
(30) edge["${\scriptstyle }$", pos=0.5, fore, black,->, curve={ratio=-0.6}, ] (29) 
(27) edge["${\scriptstyle }$", pos=0.5, fore, black,->, curve={ratio=-0.6}, ] (29) 
(30) edge["${\scriptstyle }$"', pos=0.5, fore, black,->, curve={ratio=0.6}, ] (28) 
(2) edge["${\scriptstyle }$", pos=0.5, fore, blue,->, cell=0.2, ] (37) 
(1) edge["${\scriptstyle }$", pos=0.5, fore, black,->, cell=0.2, curve={ratio=-0.4}, ] (36) 
(1) edge["${\scriptstyle }$", pos=0.5, fore, blue,->, cell=0.2, curve={ratio=0.1}, ] (36) 
(8) edge["${\scriptstyle }$", pos=0.5, fore, black,->, cell=0.2, ] (43) 
(7) edge["${\scriptstyle }$", pos=0.5, fore, blue,->, cell=0.2, curve={ratio=-0.4}, ] (42) 
(7) edge["${\scriptstyle }$", pos=0.5, fore, black,->, cell=0.2, curve={ratio=0.1}, ] (42) 
(7) edge["${\scriptstyle }$", pos=0.5, fore, red,->, cell=0.2, ] (44) 
(16) edge["${\scriptstyle }$", pos=0.5, fore, black,->, cell=0.2, curve={ratio=0.2}, ] (51) 
(15) edge["${\scriptstyle }$", pos=0.5, fore, black,->, cell=0.2, curve={ratio=-0.4}, ] (50) 
(15) edge["${\scriptstyle }$", pos=0.5, fore, black,->, cell=0.2, curve={ratio=0.1}, ] (50) 
(15) edge["${\scriptstyle }$", pos=0.5, fore, black,->, cell=0.2, curve={ratio=-0.2}, ] (52) 
(15) edge["${\scriptstyle }$", pos=0.5, fore, red,->, cell=0.2, curve={ratio=0.4}, ] (53) 
(16) edge["${\scriptstyle }$", pos=0.5, fore, blue,->, cell=0.2, curve={ratio=-0.30000000000000004}, ] (50) 
(16) edge["${\scriptstyle }$", pos=0.5, fore, orange,->, cell=0.2, curve={ratio=0.20000000000000004}, ] (50) 
(22) edge["${\scriptstyle }$", pos=0.5, fore, black,->, cell=0.2, curve={ratio=0.2}, ] (60) 
(21) edge["${\scriptstyle }$", pos=0.5, fore, black,->, cell=0.2, curve={ratio=-0.4}, ] (59) 
(21) edge["${\scriptstyle }$", pos=0.5, fore, black,->, cell=0.2, curve={ratio=0.1}, ] (59) 
(21) edge["${\scriptstyle }$", pos=0.5, fore, blue,->, cell=0.2, curve={ratio=-0.2}, ] (61) 
(22) edge["${\scriptstyle }$", pos=0.5, fore, black,->, cell=0.2, curve={ratio=-0.2}, ] (59) 
(21) edge["${\scriptstyle }$", pos=0.5, fore, blue,->, cell=0.2, curve={ratio=0.4}, ] (61) 
(22) edge["${\scriptstyle }$", pos=0.5, fore, black,->, cell=0.2, curve={ratio=0.2}, ] (59) 
(28) edge["${\scriptstyle }$", pos=0.5, fore, black,->, cell=0.2, curve={ratio=0.2}, ] (67) 
(27) edge["${\scriptstyle }$", pos=0.5, fore, black,->, cell=0.2, curve={ratio=-0.4}, ] (66) 
(27) edge["${\scriptstyle }$", pos=0.5, fore, black,->, cell=0.2, curve={ratio=0.1}, ] (66) 
(27) edge["${\scriptstyle }$", pos=0.5, fore, black,->, cell=0.2, curve={ratio=-0.2}, ] (68) 
(28) edge["${\scriptstyle }$", pos=0.5, fore, black,->, cell=0.2, curve={ratio=-0.2}, ] (66) 
(27) edge["${\scriptstyle }$", pos=0.5, fore, black,->, cell=0.2, curve={ratio=0.4}, ] (68) 
(28) edge["${\scriptstyle }$", pos=0.5, fore, black,->, cell=0.2, curve={ratio=0.2}, ] (66) 
(8) edge["${\scriptstyle }$", pos=0.5, fore, blue,->, cell=0.2, ] (43) 
(8) edge["${\scriptstyle }$", pos=0.5, fore, red,->, cell=0.2, curve={ratio=-0.2}, ] (42) 
; 
\end{tikzpicture}
% END OF GENERATED LATEX
\]
\paragraph{A version with a hide primitive}
\pagebreak
\[
% YADE DIAGRAM compo-mono-ws-2-hide.yade
% GENERATED LATEX
\begin{tikzpicture}[every node/.style={outer sep=0pt,anchor=base,text height=1.2ex, text depth=0.25ex}] 
\node[inner sep=5pt] (0) at (9.571428571428571em, -1.0863095238095237em) {$ $} ; 
\node[inner sep=5pt] (1) at (11.904761904761905em, -7.142857142857143em) {$\textcolor{black}{\bullet}$} ; 
\node[inner sep=5pt] (2) at (16.666666666666668em, -7.142857142857143em) {$\textcolor{black}{\bullet}$} ; 
\node[inner sep=5pt] (3) at (21.428571428571427em, -7.142857142857143em) {$\textcolor{black}{\bullet}$} ; 
\node[inner sep=5pt] (4) at (23.738095238095237em, -2.3720238095238093em) {$ $} ; 
\node[inner sep=5pt] (5) at (23.785714285714285em, -10.276785714285714em) {$\eqref{eq:comp-assoc}$} ; 
\node[inner sep=5pt] (6) at (7.142857142857143em, -7.142857142857143em) {$\bullet$} ; 
\node[inner sep=5pt] (7) at (30.952380952380953em, -7.142857142857143em) {$\textcolor{black}{\bullet}$} ; 
\node[inner sep=5pt] (8) at (35.714285714285715em, -7.142857142857143em) {$\textcolor{black}{\bullet}$} ; 
\node[inner sep=5pt] (9) at (40.476190476190474em, -7.142857142857143em) {$\textcolor{black}{\bullet}$} ; 
\node[inner sep=5pt] (10) at (26.19047619047619em, -7.142857142857143em) {$\bullet$} ; 
\node[inner sep=5pt] (11) at (42.833333333333336em, -2.6577380952380953em) {$ $} ; 
\node[inner sep=5pt] (12) at (42.88095238095238em, -9.75297619047619em) {$Hide$} ; 
\node[inner sep=5pt] (13) at (23.738095238095237em, -12.276785714285714em) {$ $} ; 
\node[inner sep=5pt] (14) at (42.92857142857143em, -11.657738095238095em) {$ $} ; 
\node[inner sep=5pt] (15) at (23.88095238095238em, -26.800595238095237em) {$ $} ; 
\node[inner sep=5pt] (16) at (23.928571428571427em, -34.705357142857146em) {$\eqref{eq:rew-mono-fo}$} ; 
\node[inner sep=5pt] (17) at (30.952380952380953em, -30.952380952380953em) {$\textcolor{black}{\bullet}$} ; 
\node[inner sep=5pt] (18) at (35.714285714285715em, -30.952380952380953em) {$\textcolor{black}{\bullet}$} ; 
\node[inner sep=5pt] (19) at (40.476190476190474em, -30.952380952380953em) {$\textcolor{black}{\bullet}$} ; 
\node[inner sep=5pt] (20) at (26.19047619047619em, -30.952380952380953em) {$\bullet$} ; 
\node[inner sep=5pt] (21) at (42.88095238095238em, -26.419642857142858em) {$ $} ; 
\node[inner sep=5pt] (22) at (42.92857142857143em, -34.32440476190476em) {$Hide$} ; 
\node[inner sep=5pt] (23) at (30.952380952380953em, -40.476190476190474em) {$\textcolor{black}{\bullet}$} ; 
\node[inner sep=5pt] (24) at (35.714285714285715em, -40.476190476190474em) {$\textcolor{black}{\bullet}$} ; 
\node[inner sep=5pt] (25) at (26.19047619047619em, -40.476190476190474em) {$\bullet$} ; 
\node[inner sep=5pt] (26) at (11.904761904761905em, -16.666666666666668em) {$\textcolor{black}{\bullet}$} ; 
\node[inner sep=5pt] (27) at (16.666666666666668em, -16.666666666666668em) {$\textcolor{black}{\bullet}$} ; 
\node[inner sep=5pt] (28) at (21.428571428571427em, -16.666666666666668em) {$\textcolor{black}{\bullet}$} ; 
\node[inner sep=5pt] (29) at (7.142857142857143em, -16.666666666666668em) {$\bullet$} ; 
\node[inner sep=5pt] (30) at (30.952380952380953em, -16.666666666666668em) {$\textcolor{black}{\bullet}$} ; 
\node[inner sep=5pt] (31) at (35.714285714285715em, -16.666666666666668em) {$\textcolor{black}{\bullet}$} ; 
\node[inner sep=5pt] (32) at (40.476190476190474em, -16.666666666666668em) {$\textcolor{black}{\bullet}$} ; 
\node[inner sep=5pt] (33) at (26.19047619047619em, -16.666666666666668em) {$\bullet$} ; 
\node[inner sep=5pt] (34) at (23.976190476190474em, -13.324404761904763em) {$ $} ; 
\node[inner sep=5pt] (35) at (24.023809523809526em, -21.229166666666668em) {$\eqref{eq:comp-assoc}$} ; 
\node[inner sep=5pt] (36) at (42.92857142857143em, -13.919642857142858em) {$ $} ; 
\node[inner sep=5pt] (37) at (42.976190476190474em, -21.014880952380953em) {$Hide$} ; 
\node[inner sep=5pt] (38) at (11.904761904761905em, -30.952380952380953em) {$\textcolor{black}{\bullet}$} ; 
\node[inner sep=5pt] (39) at (16.666666666666668em, -30.952380952380953em) {$\textcolor{black}{\bullet}$} ; 
\node[inner sep=5pt] (40) at (21.428571428571427em, -30.952380952380953em) {$\textcolor{black}{\bullet}$} ; 
\node[inner sep=5pt] (41) at (7.142857142857143em, -30.952380952380953em) {$\bullet$} ; 
\node[inner sep=5pt] (42) at (23.88095238095238em, -38.80059523809524em) {$ $} ; 
\node[inner sep=5pt] (43) at (23.928571428571427em, -46.705357142857146em) {$\eqref{eq:rew-mono-fo}$} ; 
\node[inner sep=5pt] (44) at (11.904761904761905em, -40.476190476190474em) {$\textcolor{black}{\bullet}$} ; 
\node[inner sep=5pt] (45) at (16.666666666666668em, -40.476190476190474em) {$\textcolor{black}{\bullet}$} ; 
\node[inner sep=5pt] (46) at (7.142857142857143em, -40.476190476190474em) {$\bullet$} ; 
\path 
(1) to[fore, blue,->, into, ] node[coordinate](47){} (2) 
(2) to[fore, blue,->, into, ] node[coordinate](48){} (3) 
(4) to[fore, black,-,] node[coordinate](49){} (5) 
(6) to[fore, blue,->, curve={ratio=0.2}, ] node[coordinate](50){} (1) 
(6) to[fore, black,->, curve={ratio=-0.30000000000000004}, ] node[coordinate](51){} (1) 
(6) to[fore, blue,->, curve={ratio=-0.6}, ] node[coordinate](52){} (3) 
(1) to[fore, blue,->, curve={ratio=-0.5}, ] node[coordinate](53){} (3) 
(7) to[fore, black,->, into, ] node[coordinate](54){} (8) 
(8) to[fore, black,->, into, ] node[coordinate](55){} (9) 
(10) to[fore, black,->, curve={ratio=0.2}, ] node[coordinate](56){} (7) 
(10) to[fore, black,->, curve={ratio=-0.30000000000000004}, ] node[coordinate](57){} (7) 
(10) to[fore, black,->, curve={ratio=-0.6}, ] node[coordinate](58){} (9) 
(7) to[fore, black,->, curve={ratio=-0.5}, ] node[coordinate](59){} (9) 
(10) to[fore, red,->, curve={ratio=0.6}, ] node[coordinate](60){} (8) 
(11) to[fore, black,-,] node[coordinate](61){} (12) 
(15) to[fore, black,-,] node[coordinate](62){} (16) 
(17) to[fore, black,->, into, ] node[coordinate](63){} (18) 
(18) to[fore, blue,->, into, ] node[coordinate](64){} (19) 
(20) to[fore, black,->, curve={ratio=0.2}, ] node[coordinate](65){} (17) 
(20) to[fore, black,->, curve={ratio=-0.30000000000000004}, ] node[coordinate](66){} (17) 
(20) to[fore, blue,->, curve={ratio=-0.6}, ] node[coordinate](67){} (19) 
(20) to[fore, black,->, curve={ratio=0.6}, ] node[coordinate](68){} (18) 
(21) to[fore, black,-,] node[coordinate](69){} (22) 
(23) to[fore, black,->, into, ] node[coordinate](70){} (24) 
(25) to[fore, black,->, ] node[coordinate](71){} (23) 
(25) to[fore, black,->, curve={ratio=0.6}, ] node[coordinate](72){} (24) 
(26) to[fore, blue,->, into, ] node[coordinate](73){} (27) 
(27) to[fore, blue,->, into, ] node[coordinate](74){} (28) 
(29) to[fore, black,->, curve={ratio=0.2}, ] node[coordinate](75){} (26) 
(29) to[fore, blue,->, curve={ratio=-0.30000000000000004}, ] node[coordinate](76){} (26) 
(29) to[fore, blue,->, curve={ratio=-0.6}, ] node[coordinate](77){} (28) 
(26) to[fore, blue,->, curve={ratio=-0.5}, ] node[coordinate](78){} (28) 
(29) to[fore, black,->, curve={ratio=0.6}, ] node[coordinate](79){} (27) 
(30) to[fore, black,->, into, ] node[coordinate](80){} (31) 
(31) to[fore, black,->, into, ] node[coordinate](81){} (32) 
(33) to[fore, black,->, curve={ratio=0.2}, ] node[coordinate](82){} (30) 
(33) to[fore, black,->, curve={ratio=-0.30000000000000004}, ] node[coordinate](83){} (30) 
(33) to[fore, black,->, curve={ratio=-0.6}, ] node[coordinate](84){} (32) 
(30) to[fore, blue,->, curve={ratio=-0.6}, ] node[coordinate](85){} (32) 
(33) to[fore, black,->, curve={ratio=0.6}, ] node[coordinate](86){} (31) 
(33) to[fore, red,->, curve={ratio=0.7999999999999999}, ] node[coordinate](87){} (31) 
(34) to[fore, black,-,] node[coordinate](88){} (35) 
(36) to[fore, black,-,] node[coordinate](89){} (37) 
(38) to[fore, black,->, into, ] node[coordinate](90){} (39) 
(39) to[fore, blue,->, into, ] node[coordinate](91){} (40) 
(41) to[fore, black,->, curve={ratio=0.2}, ] node[coordinate](92){} (38) 
(41) to[fore, black,->, curve={ratio=-0.30000000000000004}, ] node[coordinate](93){} (38) 
(41) to[fore, blue,->, curve={ratio=-0.6}, ] node[coordinate](94){} (40) 
(41) to[fore, blue,->, curve={ratio=0.6}, ] node[coordinate](95){} (39) 
(41) to[fore, blue,->, curve={ratio=0.7999999999999999}, ] node[coordinate](96){} (39) 
(42) to[fore, black,-,] node[coordinate](97){} (43) 
(44) to[fore, blue,->, into, ] node[coordinate](98){} (45) 
(46) to[fore, blue,->, curve={ratio=0.2}, ] node[coordinate](99){} (44) 
(46) to[fore, blue,->, curve={ratio=-0.30000000000000004}, ] node[coordinate](100){} (44) 
(46) to[fore, blue,->, curve={ratio=0.6}, ] node[coordinate](101){} (45) 
(2) to[fore, blue,->, cell=0.2, ] node[coordinate](102){} (53) 
(1) to[fore, black,->, cell=0.2, curve={ratio=-0.4}, ] node[coordinate](103){} (52) 
(1) to[fore, blue,->, cell=0.2, curve={ratio=0.1}, ] node[coordinate](104){} (52) 
(8) to[fore, black,->, cell=0.2, ] node[coordinate](105){} (59) 
(7) to[fore, blue,->, cell=0.2, curve={ratio=-0.4}, ] node[coordinate](106){} (58) 
(7) to[fore, black,->, cell=0.2, curve={ratio=0.1}, ] node[coordinate](107){} (58) 
(7) to[fore, red,->, cell=0.2, ] node[coordinate](108){} (60) 
(17) to[fore, black,->, cell=0.2, curve={ratio=-0.2}, ] node[coordinate](109){} (68) 
(18) to[fore, blue,->, cell=0.2, curve={ratio=-0.2}, ] node[coordinate](110){} (67) 
(17) to[fore, black,->, cell=0.2, curve={ratio=0.4}, ] node[coordinate](111){} (68) 
(18) to[fore, blue,->, cell=0.2, curve={ratio=0.2}, ] node[coordinate](112){} (67) 
(23) to[fore, black,->, cell=0.2, curve={ratio=-0.2}, ] node[coordinate](113){} (72) 
(23) to[fore, black,->, cell=0.2, curve={ratio=0.4}, ] node[coordinate](114){} (72) 
(8) to[fore, black,->, cell=0.2, ] node[coordinate](115){} (59) 
(8) to[fore, red,->, cell=0.2, curve={ratio=-0.2}, ] node[coordinate](116){} (58) 
(27) to[fore, black,->, cell=0.2, ] node[coordinate](117){} (78) 
(26) to[fore, blue,->, cell=0.2, curve={ratio=-0.1}, ] node[coordinate](118){} (77) 
(26) to[fore, black,->, cell=0.2, ] node[coordinate](119){} (79) 
(27) to[fore, blue,->, cell=0.2, ] node[coordinate](120){} (78) 
(27) to[fore, black,->, cell=0.2, curve={ratio=-0.2}, ] node[coordinate](121){} (77) 
(31) to[fore, blue,->, cell=0.2, curve={ratio=0.2}, ] node[coordinate](122){} (85) 
(30) to[fore, blue,->, cell=0.2, curve={ratio=-0.4}, ] node[coordinate](123){} (84) 
(30) to[fore, black,->, cell=0.2, curve={ratio=-0.2}, ] node[coordinate](124){} (86) 
(30) to[fore, red,->, cell=0.2, curve={ratio=0.4}, ] node[coordinate](125){} (87) 
(31) to[fore, black,->, cell=0.2, curve={ratio=-0.30000000000000004}, ] node[coordinate](126){} (84) 
(31) to[fore, red,->, cell=0.2, curve={ratio=0.20000000000000004}, ] node[coordinate](127){} (84) 
(38) to[fore, black,->, cell=0.2, curve={ratio=-0.2}, ] node[coordinate](128){} (95) 
(38) to[fore, black,->, cell=0.2, curve={ratio=0.4}, ] node[coordinate](129){} (96) 
(39) to[fore, blue,->, cell=0.2, curve={ratio=-0.30000000000000004}, ] node[coordinate](130){} (94) 
(39) to[fore, blue,->, cell=0.2, curve={ratio=0.20000000000000004}, ] node[coordinate](131){} (94) 
(44) to[fore, blue,->, cell=0.2, curve={ratio=-0.2}, ] node[coordinate](132){} (101) 
(44) to[fore, blue,->, cell=0.2, curve={ratio=0.4}, ] node[coordinate](133){} (101) 
; 
\path[->] 
(1) edge["${\scriptstyle }$", pos=0.5, fore, blue,->, into, ] (2) 
(2) edge["${\scriptstyle }$", pos=0.5, fore, blue,->, into, ] (3) 
(4) edge["${\scriptstyle }$", pos=0.5, fore, black,-,] (5) 
(6) edge["${\scriptstyle }$", pos=0.5, fore, blue,->, curve={ratio=0.2}, ] (1) 
(6) edge["${\scriptstyle }$", pos=0.5, fore, black,->, curve={ratio=-0.30000000000000004}, ] (1) 
(6) edge["${\scriptstyle }$", pos=0.5, fore, blue,->, curve={ratio=-0.6}, ] (3) 
(1) edge["${\scriptstyle }$", pos=0.5, fore, blue,->, curve={ratio=-0.5}, ] (3) 
(7) edge["${\scriptstyle }$", pos=0.5, fore, black,->, into, ] (8) 
(8) edge["${\scriptstyle }$", pos=0.5, fore, black,->, into, ] (9) 
(10) edge["${\scriptstyle }$", pos=0.5, fore, black,->, curve={ratio=0.2}, ] (7) 
(10) edge["${\scriptstyle }$", pos=0.5, fore, black,->, curve={ratio=-0.30000000000000004}, ] (7) 
(10) edge["${\scriptstyle }$", pos=0.5, fore, black,->, curve={ratio=-0.6}, ] (9) 
(7) edge["${\scriptstyle }$", pos=0.5, fore, black,->, curve={ratio=-0.5}, ] (9) 
(10) edge["${\scriptstyle }$"', pos=0.5, fore, red,->, curve={ratio=0.6}, ] (8) 
(11) edge["${\scriptstyle }$", pos=0.5, fore, black,-,] (12) 
(15) edge["${\scriptstyle }$", pos=0.5, fore, black,-,] (16) 
(17) edge["${\scriptstyle }$", pos=0.5, fore, black,->, into, ] (18) 
(18) edge["${\scriptstyle }$", pos=0.5, fore, blue,->, into, ] (19) 
(20) edge["${\scriptstyle }$", pos=0.5, fore, black,->, curve={ratio=0.2}, ] (17) 
(20) edge["${\scriptstyle }$", pos=0.5, fore, black,->, curve={ratio=-0.30000000000000004}, ] (17) 
(20) edge["${\scriptstyle }$", pos=0.5, fore, blue,->, curve={ratio=-0.6}, ] (19) 
(20) edge["${\scriptstyle }$"', pos=0.5, fore, black,->, curve={ratio=0.6}, ] (18) 
(21) edge["${\scriptstyle }$", pos=0.5, fore, black,-,] (22) 
(23) edge["${\scriptstyle }$", pos=0.5, fore, black,->, into, ] (24) 
(25) edge["${\scriptstyle }$", pos=0.5, fore, black,->, ] (23) 
(25) edge["${\scriptstyle }$"', pos=0.5, fore, black,->, curve={ratio=0.6}, ] (24) 
(26) edge["${\scriptstyle }$", pos=0.5, fore, blue,->, into, ] (27) 
(27) edge["${\scriptstyle }$", pos=0.5, fore, blue,->, into, ] (28) 
(29) edge["${\scriptstyle }$", pos=0.5, fore, black,->, curve={ratio=0.2}, ] (26) 
(29) edge["${\scriptstyle }$", pos=0.5, fore, blue,->, curve={ratio=-0.30000000000000004}, ] (26) 
(29) edge["${\scriptstyle }$", pos=0.5, fore, blue,->, curve={ratio=-0.6}, ] (28) 
(26) edge["${\scriptstyle }$", pos=0.5, fore, blue,->, curve={ratio=-0.5}, ] (28) 
(29) edge["${\scriptstyle }$"', pos=0.5, fore, black,->, curve={ratio=0.6}, ] (27) 
(30) edge["${\scriptstyle }$", pos=0.5, fore, black,->, into, ] (31) 
(31) edge["${\scriptstyle }$", pos=0.5, fore, black,->, into, ] (32) 
(33) edge["${\scriptstyle }$", pos=0.5, fore, black,->, curve={ratio=0.2}, ] (30) 
(33) edge["${\scriptstyle }$", pos=0.5, fore, black,->, curve={ratio=-0.30000000000000004}, ] (30) 
(33) edge["${\scriptstyle }$", pos=0.5, fore, black,->, curve={ratio=-0.6}, ] (32) 
(30) edge["${\scriptstyle }$", pos=0.5, fore, blue,->, curve={ratio=-0.6}, ] (32) 
(33) edge["${\scriptstyle }$"', pos=0.5, fore, black,->, curve={ratio=0.6}, ] (31) 
(33) edge["${\scriptstyle }$", pos=0.5, fore, red,->, curve={ratio=0.7999999999999999}, ] (31) 
(34) edge["${\scriptstyle }$", pos=0.5, fore, black,-,] (35) 
(36) edge["${\scriptstyle }$", pos=0.5, fore, black,-,] (37) 
(38) edge["${\scriptstyle }$", pos=0.5, fore, black,->, into, ] (39) 
(39) edge["${\scriptstyle }$", pos=0.5, fore, blue,->, into, ] (40) 
(41) edge["${\scriptstyle }$", pos=0.5, fore, black,->, curve={ratio=0.2}, ] (38) 
(41) edge["${\scriptstyle }$", pos=0.5, fore, black,->, curve={ratio=-0.30000000000000004}, ] (38) 
(41) edge["${\scriptstyle }$", pos=0.5, fore, blue,->, curve={ratio=-0.6}, ] (40) 
(41) edge["${\scriptstyle }$"', pos=0.5, fore, blue,->, curve={ratio=0.6}, ] (39) 
(41) edge["${\scriptstyle }$", pos=0.5, fore, blue,->, curve={ratio=0.7999999999999999}, ] (39) 
(42) edge["${\scriptstyle }$", pos=0.5, fore, black,-,] (43) 
(44) edge["${\scriptstyle }$", pos=0.5, fore, blue,->, into, ] (45) 
(46) edge["${\scriptstyle }$", pos=0.5, fore, blue,->, curve={ratio=0.2}, ] (44) 
(46) edge["${\scriptstyle }$", pos=0.5, fore, blue,->, curve={ratio=-0.30000000000000004}, ] (44) 
(46) edge["${\scriptstyle }$"', pos=0.5, fore, blue,->, curve={ratio=0.6}, ] (45) 
(2) edge["${\scriptstyle }$", pos=0.5, fore, blue,->, cell=0.2, ] (53) 
(1) edge["${\scriptstyle }$", pos=0.5, fore, black,->, cell=0.2, curve={ratio=-0.4}, ] (52) 
(1) edge["${\scriptstyle }$", pos=0.5, fore, blue,->, cell=0.2, curve={ratio=0.1}, ] (52) 
(8) edge["${\scriptstyle }$", pos=0.5, fore, black,->, cell=0.2, ] (59) 
(7) edge["${\scriptstyle }$", pos=0.5, fore, blue,->, cell=0.2, curve={ratio=-0.4}, ] (58) 
(7) edge["${\scriptstyle }$", pos=0.5, fore, black,->, cell=0.2, curve={ratio=0.1}, ] (58) 
(7) edge["${\scriptstyle }$", pos=0.5, fore, red,->, cell=0.2, ] (60) 
(17) edge["${\scriptstyle }$", pos=0.5, fore, black,->, cell=0.2, curve={ratio=-0.2}, ] (68) 
(18) edge["${\scriptstyle }$", pos=0.5, fore, blue,->, cell=0.2, curve={ratio=-0.2}, ] (67) 
(17) edge["${\scriptstyle }$", pos=0.5, fore, black,->, cell=0.2, curve={ratio=0.4}, ] (68) 
(18) edge["${\scriptstyle }$", pos=0.5, fore, blue,->, cell=0.2, curve={ratio=0.2}, ] (67) 
(23) edge["${\scriptstyle }$", pos=0.5, fore, black,->, cell=0.2, curve={ratio=-0.2}, ] (72) 
(23) edge["${\scriptstyle }$", pos=0.5, fore, black,->, cell=0.2, curve={ratio=0.4}, ] (72) 
(8) edge["${\scriptstyle }$", pos=0.5, fore, black,->, cell=0.2, ] (59) 
(8) edge["${\scriptstyle }$", pos=0.5, fore, red,->, cell=0.2, curve={ratio=-0.2}, ] (58) 
(27) edge["${\scriptstyle }$", pos=0.5, fore, black,->, cell=0.2, ] (78) 
(26) edge["${\scriptstyle }$", pos=0.5, fore, blue,->, cell=0.2, curve={ratio=-0.1}, ] (77) 
(26) edge["${\scriptstyle }$", pos=0.5, fore, black,->, cell=0.2, ] (79) 
(27) edge["${\scriptstyle }$", pos=0.5, fore, blue,->, cell=0.2, ] (78) 
(27) edge["${\scriptstyle }$", pos=0.5, fore, black,->, cell=0.2, curve={ratio=-0.2}, ] (77) 
(31) edge["${\scriptstyle }$", pos=0.5, fore, blue,->, cell=0.2, curve={ratio=0.2}, ] (85) 
(30) edge["${\scriptstyle }$", pos=0.5, fore, blue,->, cell=0.2, curve={ratio=-0.4}, ] (84) 
(30) edge["${\scriptstyle }$", pos=0.5, fore, black,->, cell=0.2, curve={ratio=-0.2}, ] (86) 
(30) edge["${\scriptstyle }$", pos=0.5, fore, red,->, cell=0.2, curve={ratio=0.4}, ] (87) 
(31) edge["${\scriptstyle }$", pos=0.5, fore, black,->, cell=0.2, curve={ratio=-0.30000000000000004}, ] (84) 
(31) edge["${\scriptstyle }$", pos=0.5, fore, red,->, cell=0.2, curve={ratio=0.20000000000000004}, ] (84) 
(38) edge["${\scriptstyle }$", pos=0.5, fore, black,->, cell=0.2, curve={ratio=-0.2}, ] (95) 
(38) edge["${\scriptstyle }$", pos=0.5, fore, black,->, cell=0.2, curve={ratio=0.4}, ] (96) 
(39) edge["${\scriptstyle }$", pos=0.5, fore, blue,->, cell=0.2, curve={ratio=-0.30000000000000004}, ] (94) 
(39) edge["${\scriptstyle }$", pos=0.5, fore, blue,->, cell=0.2, curve={ratio=0.20000000000000004}, ] (94) 
(44) edge["${\scriptstyle }$", pos=0.5, fore, blue,->, cell=0.2, curve={ratio=-0.2}, ] (101) 
(44) edge["${\scriptstyle }$", pos=0.5, fore, blue,->, cell=0.2, curve={ratio=0.4}, ] (101) 
; 
\end{tikzpicture}
% END OF GENERATED LATEX
\]
\paragraph{A version with hide and focus primitives}
The idea is that we keep all the elements that depend on the selected elements, and hide the rest.
\pagebreak
\[
% YADE DIAGRAM compo-mono-ws-2-focus.yade
% GENERATED LATEX
\begin{tikzpicture}[every node/.style={outer sep=0pt,anchor=base,text height=1.2ex, text depth=0.25ex}] 
\node[inner sep=5pt] (0) at (9.571428571428571em, -1.0863095238095237em) {$ $} ; 
\node[inner sep=5pt] (1) at (11.904761904761905em, -7.142857142857143em) {$\textcolor{black}{\bullet}$} ; 
\node[inner sep=5pt] (2) at (16.666666666666668em, -7.142857142857143em) {$\textcolor{black}{\bullet}$} ; 
\node[inner sep=5pt] (3) at (21.428571428571427em, -7.142857142857143em) {$\textcolor{black}{\bullet}$} ; 
\node[inner sep=5pt] (4) at (7.142857142857143em, -7.142857142857143em) {$\bullet$} ; 
\node[inner sep=5pt] (5) at (30.952380952380953em, -7.142857142857143em) {$\textcolor{black}{\bullet}$} ; 
\node[inner sep=5pt] (6) at (35.714285714285715em, -7.142857142857143em) {$\textcolor{black}{\bullet}$} ; 
\node[inner sep=5pt] (7) at (40.476190476190474em, -7.142857142857143em) {$\textcolor{black}{\bullet}$} ; 
\node[inner sep=5pt] (8) at (26.19047619047619em, -7.142857142857143em) {$\bullet$} ; 
\node[inner sep=5pt] (9) at (42.833333333333336em, -2.6577380952380953em) {$ $} ; 
\node[inner sep=5pt] (10) at (42.88095238095238em, -9.75297619047619em) {$\eqref{eq:comp-assoc}$} ; 
\node[inner sep=5pt] (11) at (23.738095238095237em, -12.276785714285714em) {$ $} ; 
\node[inner sep=5pt] (12) at (42.92857142857143em, -11.657738095238095em) {$ $} ; 
\node[inner sep=5pt] (13) at (23.88095238095238em, -36.32440476190476em) {$ $} ; 
\node[inner sep=5pt] (14) at (23.928571428571427em, -44.229166666666664em) {$Hide$} ; 
\node[inner sep=5pt] (15) at (11.904761904761905em, -50em) {$\textcolor{black}{\bullet}$} ; 
\node[inner sep=5pt] (16) at (16.666666666666668em, -50em) {$\textcolor{black}{\bullet}$} ; 
\node[inner sep=5pt] (17) at (21.428571428571427em, -50em) {$\textcolor{black}{\bullet}$} ; 
\node[inner sep=5pt] (18) at (7.142857142857143em, -50em) {$\bullet$} ; 
\node[inner sep=5pt] (19) at (42.88095238095238em, -35.94345238095238em) {$ $} ; 
\node[inner sep=5pt] (20) at (42.92857142857143em, -43.848214285714285em) {$\eqref{eq:rew-mono-fo}$} ; 
\node[inner sep=5pt] (21) at (11.904761904761905em, -16.666666666666668em) {$\textcolor{black}{\bullet}$} ; 
\node[inner sep=5pt] (22) at (16.666666666666668em, -16.666666666666668em) {$\textcolor{black}{\bullet}$} ; 
\node[inner sep=5pt] (23) at (21.428571428571427em, -16.666666666666668em) {$\textcolor{black}{\bullet}$} ; 
\node[inner sep=5pt] (24) at (7.142857142857143em, -16.666666666666668em) {$\bullet$} ; 
\node[inner sep=5pt] (25) at (23.976190476190474em, -13.324404761904763em) {$ $} ; 
\node[inner sep=5pt] (26) at (24.023809523809526em, -21.229166666666668em) {$Unfocus$} ; 
\node[inner sep=5pt] (27) at (42.92857142857143em, -13.919642857142858em) {$ $} ; 
\node[inner sep=5pt] (28) at (42.976190476190474em, -21.014880952380953em) {$Focus$} ; 
\node[inner sep=5pt] (29) at (23.785714285714285em, -2.0148809523809526em) {$ $} ; 
\node[inner sep=5pt] (30) at (23.833333333333332em, -9.110119047619047em) {$Focus$} ; 
\node[inner sep=5pt] (31) at (30.952380952380953em, -16.666666666666668em) {$\textcolor{black}{\bullet}$} ; 
\node[inner sep=5pt] (32) at (35.714285714285715em, -16.666666666666668em) {$\textcolor{black}{\bullet}$} ; 
\node[inner sep=5pt] (33) at (40.476190476190474em, -16.666666666666668em) {$\textcolor{black}{\bullet}$} ; 
\node[inner sep=5pt] (34) at (26.19047619047619em, -16.666666666666668em) {$\bullet$} ; 
\node[inner sep=5pt] (35) at (11.904761904761905em, -30.952380952380953em) {$\textcolor{black}{\bullet}$} ; 
\node[inner sep=5pt] (36) at (16.666666666666668em, -30.952380952380953em) {$\textcolor{black}{\bullet}$} ; 
\node[inner sep=5pt] (37) at (21.428571428571427em, -30.952380952380953em) {$\textcolor{black}{\bullet}$} ; 
\node[inner sep=5pt] (38) at (7.142857142857143em, -30.952380952380953em) {$\bullet$} ; 
\node[inner sep=5pt] (39) at (24.023809523809526em, -24.538690476190474em) {$ $} ; 
\node[inner sep=5pt] (40) at (24.071428571428573em, -34.467261904761905em) {$\eqref{eq:comp-assoc}$} ; 
\node[inner sep=5pt] (41) at (30.952380952380953em, -30.952380952380953em) {$\textcolor{black}{\bullet}$} ; 
\node[inner sep=5pt] (42) at (35.714285714285715em, -30.952380952380953em) {$\textcolor{black}{\bullet}$} ; 
\node[inner sep=5pt] (43) at (40.476190476190474em, -30.952380952380953em) {$\textcolor{black}{\bullet}$} ; 
\node[inner sep=5pt] (44) at (26.19047619047619em, -30.952380952380953em) {$\bullet$} ; 
\node[inner sep=5pt] (45) at (42.92857142857143em, -25.729166666666668em) {$ $} ; 
\node[inner sep=5pt] (46) at (42.976190476190474em, -32.82440476190476em) {$Unfocus$} ; 
\node[inner sep=5pt] (47) at (11.904761904761905em, -40.476190476190474em) {$\textcolor{black}{\bullet}$} ; 
\node[inner sep=5pt] (48) at (16.666666666666668em, -40.476190476190474em) {$\textcolor{black}{\bullet}$} ; 
\node[inner sep=5pt] (49) at (21.428571428571427em, -40.476190476190474em) {$\textcolor{black}{\bullet}$} ; 
\node[inner sep=5pt] (50) at (7.142857142857143em, -40.476190476190474em) {$\bullet$} ; 
\node[inner sep=5pt] (51) at (30.952380952380953em, -40.476190476190474em) {$\textcolor{black}{\bullet}$} ; 
\node[inner sep=5pt] (52) at (35.714285714285715em, -40.476190476190474em) {$\textcolor{black}{\bullet}$} ; 
\node[inner sep=5pt] (53) at (40.476190476190474em, -40.476190476190474em) {$\textcolor{black}{\bullet}$} ; 
\node[inner sep=5pt] (54) at (26.19047619047619em, -40.476190476190474em) {$\bullet$} ; 
\node[inner sep=5pt] (55) at (23.928571428571427em, -45.419642857142854em) {$ $} ; 
\node[inner sep=5pt] (56) at (23.976190476190474em, -53.32440476190476em) {$\eqref{eq:rew-mono-fo}$} ; 
\node[inner sep=5pt] (57) at (30.952380952380953em, -50em) {$\textcolor{black}{\bullet}$} ; 
\node[inner sep=5pt] (58) at (35.714285714285715em, -50em) {$\textcolor{black}{\bullet}$} ; 
\node[inner sep=5pt] (59) at (40.476190476190474em, -50em) {$\textcolor{black}{\bullet}$} ; 
\node[inner sep=5pt] (60) at (26.19047619047619em, -50em) {$\bullet$} ; 
\path 
(1) to[fore, black,->, into, ] node[coordinate](61){} (2) 
(2) to[fore, black,->, into, ] node[coordinate](62){} (3) 
(4) to[fore, blue,->, curve={ratio=0.2}, ] node[coordinate](63){} (1) 
(4) to[fore, black,->, curve={ratio=-0.30000000000000004}, ] node[coordinate](64){} (1) 
(4) to[fore, black,->, curve={ratio=-0.6}, ] node[coordinate](65){} (3) 
(1) to[fore, black,->, curve={ratio=-0.5}, ] node[coordinate](66){} (3) 
(5) to[fore, blue,->, into, ] node[coordinate](67){} (6) 
(6) to[fore, blue,->, into, ] node[coordinate](68){} (7) 
(8) to[fore, blue,->, curve={ratio=0.2}, ] node[coordinate](69){} (5) 
(8) to[fore, blue,->, curve={ratio=-0.6}, ] node[coordinate](70){} (7) 
(5) to[fore, blue,->, curve={ratio=-0.5}, ] node[coordinate](71){} (7) 
(9) to[fore, black,-,] node[coordinate](72){} (10) 
(13) to[fore, black,-,] node[coordinate](73){} (14) 
(15) to[fore, blue,->, into, ] node[coordinate](74){} (16) 
(16) to[fore, black,->, into, ] node[coordinate](75){} (17) 
(18) to[fore, blue,->, curve={ratio=0.2}, ] node[coordinate](76){} (15) 
(18) to[fore, blue,->, curve={ratio=-0.30000000000000004}, ] node[coordinate](77){} (15) 
(18) to[fore, black,->, curve={ratio=-0.6}, ] node[coordinate](78){} (17) 
(18) to[fore, blue,->, curve={ratio=0.6}, ] node[coordinate](79){} (16) 
(19) to[fore, black,-,] node[coordinate](80){} (20) 
(21) to[fore, black,->, into, ] node[coordinate](81){} (22) 
(22) to[fore, black,->, into, ] node[coordinate](82){} (23) 
(24) to[fore, black,->, curve={ratio=0.2}, ] node[coordinate](83){} (21) 
(24) to[fore, black,->, curve={ratio=-0.6}, ] node[coordinate](84){} (23) 
(21) to[fore, black,->, curve={ratio=-0.5}, ] node[coordinate](85){} (23) 
(24) to[fore, red,->, curve={ratio=0.6}, ] node[coordinate](86){} (22) 
(25) to[fore, black,-,] node[coordinate](87){} (26) 
(27) to[fore, black,-,] node[coordinate](88){} (28) 
(29) to[fore, black,-,] node[coordinate](89){} (30) 
(31) to[fore, black,->, into, ] node[coordinate](90){} (32) 
(32) to[fore, black,->, into, ] node[coordinate](91){} (33) 
(34) to[fore, black,->, curve={ratio=0.2}, ] node[coordinate](92){} (31) 
(34) to[fore, blue,->, curve={ratio=-0.30000000000000004}, ] node[coordinate](93){} (31) 
(34) to[fore, black,->, curve={ratio=-0.6}, ] node[coordinate](94){} (33) 
(31) to[fore, black,->, curve={ratio=-0.5}, ] node[coordinate](95){} (33) 
(34) to[fore, black,->, curve={ratio=0.7}, ] node[coordinate](96){} (32) 
(35) to[fore, blue,->, into, ] node[coordinate](97){} (36) 
(36) to[fore, blue,->, into, ] node[coordinate](98){} (37) 
(38) to[fore, blue,->, curve={ratio=-0.2}, ] node[coordinate](99){} (35) 
(38) to[fore, blue,->, curve={ratio=-0.6}, ] node[coordinate](100){} (37) 
(35) to[fore, blue,->, curve={ratio=-0.5}, ] node[coordinate](101){} (37) 
(39) to[fore, black,-,] node[coordinate](102){} (40) 
(41) to[fore, black,->, into, ] node[coordinate](103){} (42) 
(42) to[fore, black,->, into, ] node[coordinate](104){} (43) 
(44) to[fore, black,->, curve={ratio=-0.30000000000000004}, ] node[coordinate](105){} (41) 
(44) to[fore, black,->, curve={ratio=-0.6}, ] node[coordinate](106){} (43) 
(41) to[fore, black,->, curve={ratio=-0.5}, ] node[coordinate](107){} (43) 
(44) to[fore, red,->, curve={ratio=0.6}, ] node[coordinate](108){} (42) 
(45) to[fore, black,-,] node[coordinate](109){} (46) 
(47) to[fore, black,->, into, ] node[coordinate](110){} (48) 
(48) to[fore, black,->, into, ] node[coordinate](111){} (49) 
(50) to[fore, black,->, curve={ratio=0.2}, ] node[coordinate](112){} (47) 
(50) to[fore, black,->, curve={ratio=-0.30000000000000004}, ] node[coordinate](113){} (47) 
(50) to[fore, black,->, curve={ratio=-0.6}, ] node[coordinate](114){} (49) 
(47) to[fore, blue,->, curve={ratio=-0.6}, ] node[coordinate](115){} (49) 
(50) to[fore, black,->, curve={ratio=0.6}, ] node[coordinate](116){} (48) 
(50) to[fore, black,->, curve={ratio=0.7999999999999999}, ] node[coordinate](117){} (48) 
(51) to[fore, black,->, into, ] node[coordinate](118){} (52) 
(52) to[fore, blue,->, into, ] node[coordinate](119){} (53) 
(54) to[fore, black,->, curve={ratio=0.2}, ] node[coordinate](120){} (51) 
(54) to[fore, black,->, curve={ratio=-0.30000000000000004}, ] node[coordinate](121){} (51) 
(54) to[fore, blue,->, curve={ratio=-0.6}, ] node[coordinate](122){} (53) 
(54) to[fore, blue,->, curve={ratio=0.6}, ] node[coordinate](123){} (52) 
(54) to[fore, blue,->, curve={ratio=0.7999999999999999}, ] node[coordinate](124){} (52) 
(55) to[fore, black,-,] node[coordinate](125){} (56) 
(57) to[fore, black,->, into, ] node[coordinate](126){} (58) 
(58) to[fore, black,->, into, ] node[coordinate](127){} (59) 
(60) to[fore, black,->, ] node[coordinate](128){} (57) 
(60) to[fore, black,->, curve={ratio=-0.6}, ] node[coordinate](129){} (59) 
(60) to[fore, black,->, curve={ratio=0.6}, ] node[coordinate](130){} (58) 
(2) to[fore, black,->, cell=0.2, ] node[coordinate](131){} (66) 
(1) to[fore, black,->, cell=0.2, curve={ratio=-0.4}, ] node[coordinate](132){} (65) 
(1) to[fore, black,->, cell=0.2, curve={ratio=0.1}, ] node[coordinate](133){} (65) 
(6) to[fore, blue,->, cell=0.2, ] node[coordinate](134){} (71) 
(5) to[fore, blue,->, cell=0.2, curve={ratio=-0.4}, ] node[coordinate](135){} (70) 
(15) to[fore, blue,->, cell=0.2, curve={ratio=-0.2}, ] node[coordinate](136){} (79) 
(16) to[fore, black,->, cell=0.2, curve={ratio=-0.2}, ] node[coordinate](137){} (78) 
(15) to[fore, blue,->, cell=0.2, curve={ratio=0.4}, ] node[coordinate](138){} (79) 
(16) to[fore, black,->, cell=0.2, curve={ratio=0.2}, ] node[coordinate](139){} (78) 
(6) to[fore, blue,->, cell=0.2, ] node[coordinate](140){} (71) 
(22) to[fore, black,->, cell=0.2, ] node[coordinate](141){} (85) 
(21) to[fore, black,->, cell=0.2, curve={ratio=-0.1}, ] node[coordinate](142){} (84) 
(21) to[fore, red,->, cell=0.2, ] node[coordinate](143){} (86) 
(22) to[fore, black,->, cell=0.2, ] node[coordinate](144){} (85) 
(22) to[fore, red,->, cell=0.2, curve={ratio=-0.2}, ] node[coordinate](145){} (84) 
(32) to[fore, black,->, cell=0.2, ] node[coordinate](146){} (95) 
(31) to[fore, black,->, cell=0.2, curve={ratio=-0.4}, ] node[coordinate](147){} (94) 
(31) to[fore, black,->, cell=0.2, curve={ratio=0.1}, ] node[coordinate](148){} (94) 
(32) to[fore, black,->, cell=0.2, ] node[coordinate](149){} (94) 
(31) to[fore, black,->, cell=0.2, ] node[coordinate](150){} (96) 
(36) to[fore, blue,->, cell=0.2, ] node[coordinate](151){} (101) 
(35) to[fore, blue,->, cell=0.2, curve={ratio=-0.4}, ] node[coordinate](152){} (100) 
(36) to[fore, blue,->, cell=0.2, ] node[coordinate](153){} (101) 
(42) to[fore, black,->, cell=0.2, ] node[coordinate](154){} (107) 
(41) to[fore, black,->, cell=0.2, curve={ratio=-0.2}, ] node[coordinate](155){} (106) 
(41) to[fore, red,->, cell=0.2, ] node[coordinate](156){} (108) 
(42) to[fore, black,->, cell=0.2, ] node[coordinate](157){} (107) 
(42) to[fore, red,->, cell=0.2, curve={ratio=0.1}, ] node[coordinate](158){} (106) 
(48) to[fore, blue,->, cell=0.2, curve={ratio=0.2}, ] node[coordinate](159){} (115) 
(47) to[fore, blue,->, cell=0.2, curve={ratio=-0.4}, ] node[coordinate](160){} (114) 
(47) to[fore, blue,->, cell=0.2, curve={ratio=0.1}, ] node[coordinate](161){} (114) 
(47) to[fore, black,->, cell=0.2, curve={ratio=-0.2}, ] node[coordinate](162){} (116) 
(47) to[fore, black,->, cell=0.2, curve={ratio=0.4}, ] node[coordinate](163){} (117) 
(48) to[fore, black,->, cell=0.2, curve={ratio=-0.30000000000000004}, ] node[coordinate](164){} (114) 
(48) to[fore, black,->, cell=0.2, curve={ratio=0.20000000000000004}, ] node[coordinate](165){} (114) 
(51) to[fore, black,->, cell=0.2, curve={ratio=-0.2}, ] node[coordinate](166){} (123) 
(51) to[fore, black,->, cell=0.2, curve={ratio=0.4}, ] node[coordinate](167){} (124) 
(52) to[fore, blue,->, cell=0.2, curve={ratio=-0.30000000000000004}, ] node[coordinate](168){} (122) 
(52) to[fore, blue,->, cell=0.2, curve={ratio=0.20000000000000004}, ] node[coordinate](169){} (122) 
(57) to[fore, black,->, cell=0.2, curve={ratio=-0.2}, ] node[coordinate](170){} (130) 
(58) to[fore, black,->, cell=0.2, curve={ratio=-0.2}, ] node[coordinate](171){} (129) 
(57) to[fore, black,->, cell=0.2, curve={ratio=0.4}, ] node[coordinate](172){} (130) 
(58) to[fore, black,->, cell=0.2, curve={ratio=0.2}, ] node[coordinate](173){} (129) 
; 
\path[->] 
(1) edge["${\scriptstyle }$", pos=0.5, fore, black,->, into, ] (2) 
(2) edge["${\scriptstyle }$", pos=0.5, fore, black,->, into, ] (3) 
(4) edge["${\scriptstyle }$", pos=0.5, fore, blue,->, curve={ratio=0.2}, ] (1) 
(4) edge["${\scriptstyle }$", pos=0.5, fore, black,->, curve={ratio=-0.30000000000000004}, ] (1) 
(4) edge["${\scriptstyle }$", pos=0.5, fore, black,->, curve={ratio=-0.6}, ] (3) 
(1) edge["${\scriptstyle }$", pos=0.5, fore, black,->, curve={ratio=-0.5}, ] (3) 
(5) edge["${\scriptstyle }$", pos=0.5, fore, blue,->, into, ] (6) 
(6) edge["${\scriptstyle }$", pos=0.5, fore, blue,->, into, ] (7) 
(8) edge["${\scriptstyle }$", pos=0.5, fore, blue,->, curve={ratio=0.2}, ] (5) 
(8) edge["${\scriptstyle }$", pos=0.5, fore, blue,->, curve={ratio=-0.6}, ] (7) 
(5) edge["${\scriptstyle }$", pos=0.5, fore, blue,->, curve={ratio=-0.5}, ] (7) 
(9) edge["${\scriptstyle }$", pos=0.5, fore, black,-,] (10) 
(13) edge["${\scriptstyle }$", pos=0.5, fore, black,-,] (14) 
(15) edge["${\scriptstyle }$", pos=0.5, fore, blue,->, into, ] (16) 
(16) edge["${\scriptstyle }$", pos=0.5, fore, black,->, into, ] (17) 
(18) edge["${\scriptstyle }$", pos=0.5, fore, blue,->, curve={ratio=0.2}, ] (15) 
(18) edge["${\scriptstyle }$", pos=0.5, fore, blue,->, curve={ratio=-0.30000000000000004}, ] (15) 
(18) edge["${\scriptstyle }$", pos=0.5, fore, black,->, curve={ratio=-0.6}, ] (17) 
(18) edge["${\scriptstyle }$"', pos=0.5, fore, blue,->, curve={ratio=0.6}, ] (16) 
(19) edge["${\scriptstyle }$", pos=0.5, fore, black,-,] (20) 
(21) edge["${\scriptstyle }$", pos=0.5, fore, black,->, into, ] (22) 
(22) edge["${\scriptstyle }$", pos=0.5, fore, black,->, into, ] (23) 
(24) edge["${\scriptstyle }$", pos=0.5, fore, black,->, curve={ratio=0.2}, ] (21) 
(24) edge["${\scriptstyle }$", pos=0.5, fore, black,->, curve={ratio=-0.6}, ] (23) 
(21) edge["${\scriptstyle }$", pos=0.5, fore, black,->, curve={ratio=-0.5}, ] (23) 
(24) edge["${\scriptstyle }$"', pos=0.5, fore, red,->, curve={ratio=0.6}, ] (22) 
(25) edge["${\scriptstyle }$", pos=0.5, fore, black,-,] (26) 
(27) edge["${\scriptstyle }$", pos=0.5, fore, black,-,] (28) 
(29) edge["${\scriptstyle }$", pos=0.5, fore, black,-,] (30) 
(31) edge["${\scriptstyle }$", pos=0.5, fore, black,->, into, ] (32) 
(32) edge["${\scriptstyle }$", pos=0.5, fore, black,->, into, ] (33) 
(34) edge["${\scriptstyle }$", pos=0.5, fore, black,->, curve={ratio=0.2}, ] (31) 
(34) edge["${\scriptstyle }$", pos=0.5, fore, blue,->, curve={ratio=-0.30000000000000004}, ] (31) 
(34) edge["${\scriptstyle }$", pos=0.5, fore, black,->, curve={ratio=-0.6}, ] (33) 
(31) edge["${\scriptstyle }$", pos=0.5, fore, black,->, curve={ratio=-0.5}, ] (33) 
(34) edge["${\scriptstyle }$", pos=0.5, fore, black,->, curve={ratio=0.7}, ] (32) 
(35) edge["${\scriptstyle }$", pos=0.5, fore, blue,->, into, ] (36) 
(36) edge["${\scriptstyle }$", pos=0.5, fore, blue,->, into, ] (37) 
(38) edge["${\scriptstyle }$", pos=0.5, fore, blue,->, curve={ratio=-0.2}, ] (35) 
(38) edge["${\scriptstyle }$", pos=0.5, fore, blue,->, curve={ratio=-0.6}, ] (37) 
(35) edge["${\scriptstyle }$", pos=0.5, fore, blue,->, curve={ratio=-0.5}, ] (37) 
(39) edge["${\scriptstyle }$", pos=0.5, fore, black,-,] (40) 
(41) edge["${\scriptstyle }$", pos=0.5, fore, black,->, into, ] (42) 
(42) edge["${\scriptstyle }$", pos=0.5, fore, black,->, into, ] (43) 
(44) edge["${\scriptstyle }$", pos=0.5, fore, black,->, curve={ratio=-0.30000000000000004}, ] (41) 
(44) edge["${\scriptstyle }$", pos=0.5, fore, black,->, curve={ratio=-0.6}, ] (43) 
(41) edge["${\scriptstyle }$", pos=0.5, fore, black,->, curve={ratio=-0.5}, ] (43) 
(44) edge["${\scriptstyle }$"', pos=0.5, fore, red,->, curve={ratio=0.6}, ] (42) 
(45) edge["${\scriptstyle }$", pos=0.5, fore, black,-,] (46) 
(47) edge["${\scriptstyle }$", pos=0.5, fore, black,->, into, ] (48) 
(48) edge["${\scriptstyle }$", pos=0.5, fore, black,->, into, ] (49) 
(50) edge["${\scriptstyle }$", pos=0.5, fore, black,->, curve={ratio=0.2}, ] (47) 
(50) edge["${\scriptstyle }$", pos=0.5, fore, black,->, curve={ratio=-0.30000000000000004}, ] (47) 
(50) edge["${\scriptstyle }$", pos=0.5, fore, black,->, curve={ratio=-0.6}, ] (49) 
(47) edge["${\scriptstyle }$", pos=0.5, fore, blue,->, curve={ratio=-0.6}, ] (49) 
(50) edge["${\scriptstyle }$"', pos=0.5, fore, black,->, curve={ratio=0.6}, ] (48) 
(50) edge["${\scriptstyle }$", pos=0.5, fore, black,->, curve={ratio=0.7999999999999999}, ] (48) 
(51) edge["${\scriptstyle }$", pos=0.5, fore, black,->, into, ] (52) 
(52) edge["${\scriptstyle }$", pos=0.5, fore, blue,->, into, ] (53) 
(54) edge["${\scriptstyle }$", pos=0.5, fore, black,->, curve={ratio=0.2}, ] (51) 
(54) edge["${\scriptstyle }$", pos=0.5, fore, black,->, curve={ratio=-0.30000000000000004}, ] (51) 
(54) edge["${\scriptstyle }$", pos=0.5, fore, blue,->, curve={ratio=-0.6}, ] (53) 
(54) edge["${\scriptstyle }$"', pos=0.5, fore, blue,->, curve={ratio=0.6}, ] (52) 
(54) edge["${\scriptstyle }$", pos=0.5, fore, blue,->, curve={ratio=0.7999999999999999}, ] (52) 
(55) edge["${\scriptstyle }$", pos=0.5, fore, black,-,] (56) 
(57) edge["${\scriptstyle }$", pos=0.5, fore, black,->, into, ] (58) 
(58) edge["${\scriptstyle }$", pos=0.5, fore, black,->, into, ] (59) 
(60) edge["${\scriptstyle }$", pos=0.5, fore, black,->, ] (57) 
(60) edge["${\scriptstyle }$", pos=0.5, fore, black,->, curve={ratio=-0.6}, ] (59) 
(60) edge["${\scriptstyle }$"', pos=0.5, fore, black,->, curve={ratio=0.6}, ] (58) 
(2) edge["${\scriptstyle }$", pos=0.5, fore, black,->, cell=0.2, ] (66) 
(1) edge["${\scriptstyle }$", pos=0.5, fore, black,->, cell=0.2, curve={ratio=-0.4}, ] (65) 
(1) edge["${\scriptstyle }$", pos=0.5, fore, black,->, cell=0.2, curve={ratio=0.1}, ] (65) 
(6) edge["${\scriptstyle }$", pos=0.5, fore, blue,->, cell=0.2, ] (71) 
(5) edge["${\scriptstyle }$", pos=0.5, fore, blue,->, cell=0.2, curve={ratio=-0.4}, ] (70) 
(15) edge["${\scriptstyle }$", pos=0.5, fore, blue,->, cell=0.2, curve={ratio=-0.2}, ] (79) 
(16) edge["${\scriptstyle }$", pos=0.5, fore, black,->, cell=0.2, curve={ratio=-0.2}, ] (78) 
(15) edge["${\scriptstyle }$", pos=0.5, fore, blue,->, cell=0.2, curve={ratio=0.4}, ] (79) 
(16) edge["${\scriptstyle }$", pos=0.5, fore, black,->, cell=0.2, curve={ratio=0.2}, ] (78) 
(6) edge["${\scriptstyle }$", pos=0.5, fore, blue,->, cell=0.2, ] (71) 
(22) edge["${\scriptstyle }$", pos=0.5, fore, black,->, cell=0.2, ] (85) 
(21) edge["${\scriptstyle }$", pos=0.5, fore, black,->, cell=0.2, curve={ratio=-0.1}, ] (84) 
(21) edge["${\scriptstyle }$", pos=0.5, fore, red,->, cell=0.2, ] (86) 
(22) edge["${\scriptstyle }$", pos=0.5, fore, black,->, cell=0.2, ] (85) 
(22) edge["${\scriptstyle }$", pos=0.5, fore, red,->, cell=0.2, curve={ratio=-0.2}, ] (84) 
(32) edge["${\scriptstyle }$", pos=0.5, fore, black,->, cell=0.2, ] (95) 
(31) edge["${\scriptstyle }$", pos=0.5, fore, black,->, cell=0.2, curve={ratio=-0.4}, ] (94) 
(31) edge["${\scriptstyle }$", pos=0.5, fore, black,->, cell=0.2, curve={ratio=0.1}, ] (94) 
(32) edge["${\scriptstyle }$", pos=0.5, fore, black,->, cell=0.2, ] (94) 
(31) edge["${\scriptstyle }$", pos=0.5, fore, black,->, cell=0.2, ] (96) 
(36) edge["${\scriptstyle }$", pos=0.5, fore, blue,->, cell=0.2, ] (101) 
(35) edge["${\scriptstyle }$", pos=0.5, fore, blue,->, cell=0.2, curve={ratio=-0.4}, ] (100) 
(36) edge["${\scriptstyle }$", pos=0.5, fore, blue,->, cell=0.2, ] (101) 
(42) edge["${\scriptstyle }$", pos=0.5, fore, black,->, cell=0.2, ] (107) 
(41) edge["${\scriptstyle }$", pos=0.5, fore, black,->, cell=0.2, curve={ratio=-0.2}, ] (106) 
(41) edge["${\scriptstyle }$", pos=0.5, fore, red,->, cell=0.2, ] (108) 
(42) edge["${\scriptstyle }$", pos=0.5, fore, black,->, cell=0.2, ] (107) 
(42) edge["${\scriptstyle }$", pos=0.5, fore, red,->, cell=0.2, curve={ratio=0.1}, ] (106) 
(48) edge["${\scriptstyle }$", pos=0.5, fore, blue,->, cell=0.2, curve={ratio=0.2}, ] (115) 
(47) edge["${\scriptstyle }$", pos=0.5, fore, blue,->, cell=0.2, curve={ratio=-0.4}, ] (114) 
(47) edge["${\scriptstyle }$", pos=0.5, fore, blue,->, cell=0.2, curve={ratio=0.1}, ] (114) 
(47) edge["${\scriptstyle }$", pos=0.5, fore, black,->, cell=0.2, curve={ratio=-0.2}, ] (116) 
(47) edge["${\scriptstyle }$", pos=0.5, fore, black,->, cell=0.2, curve={ratio=0.4}, ] (117) 
(48) edge["${\scriptstyle }$", pos=0.5, fore, black,->, cell=0.2, curve={ratio=-0.30000000000000004}, ] (114) 
(48) edge["${\scriptstyle }$", pos=0.5, fore, black,->, cell=0.2, curve={ratio=0.20000000000000004}, ] (114) 
(51) edge["${\scriptstyle }$", pos=0.5, fore, black,->, cell=0.2, curve={ratio=-0.2}, ] (123) 
(51) edge["${\scriptstyle }$", pos=0.5, fore, black,->, cell=0.2, curve={ratio=0.4}, ] (124) 
(52) edge["${\scriptstyle }$", pos=0.5, fore, blue,->, cell=0.2, curve={ratio=-0.30000000000000004}, ] (122) 
(52) edge["${\scriptstyle }$", pos=0.5, fore, blue,->, cell=0.2, curve={ratio=0.20000000000000004}, ] (122) 
(57) edge["${\scriptstyle }$", pos=0.5, fore, black,->, cell=0.2, curve={ratio=-0.2}, ] (130) 
(58) edge["${\scriptstyle }$", pos=0.5, fore, black,->, cell=0.2, curve={ratio=-0.2}, ] (129) 
(57) edge["${\scriptstyle }$", pos=0.5, fore, black,->, cell=0.2, curve={ratio=0.4}, ] (130) 
(58) edge["${\scriptstyle }$", pos=0.5, fore, black,->, cell=0.2, curve={ratio=0.2}, ] (129) 
; 
\end{tikzpicture}
% END OF GENERATED LATEX
\]

Maybe names could help, to remember stuff...
\paragraph{A version with hide tactical}
We assume that we have a tactic \textsc{Apply-Hide} that takes as input 
a rewriting rule, and a set of elements to hide.
It will apply the rewriting rules greedily on the selected elements,
and then hide the objects.
\[
% YADE DIAGRAM compo-mono-apply-hide.yade
% GENERATED LATEX
\begin{tikzpicture}[every node/.style={outer sep=0pt,anchor=base,text height=1.2ex, text depth=0.25ex}] 
\node[inner sep=5pt] (0) at (10.46875em, -1.1881510416666667em) {$ $} ; 
\node[inner sep=5pt] (1) at (13.020833333333334em, -7.8125em) {$\textcolor{black}{\bullet}$} ; 
\node[inner sep=5pt] (2) at (18.229166666666668em, -7.8125em) {$\textcolor{black}{\bullet}$} ; 
\node[inner sep=5pt] (3) at (23.4375em, -7.8125em) {$\textcolor{black}{\bullet}$} ; 
\node[inner sep=5pt] (4) at (7.8125em, -7.8125em) {$\bullet$} ; 
\node[inner sep=5pt] (5) at (46.848958333333336em, -2.906901041666667em) {$ $} ; 
\node[inner sep=5pt] (6) at (46.90104166666667em, -10.667317708333334em) {$\eqref{eq:rew-mono-fo}$} ; 
\node[inner sep=5pt] (7) at (25.963541666666668em, -13.427734375em) {$ $} ; 
\node[inner sep=5pt] (8) at (46.953125em, -12.750651041666668em) {$ $} ; 
\node[inner sep=5pt] (9) at (26.171875em, -17.125651041666668em) {$Hide$} ; 
\node[inner sep=5pt] (10) at (13.020833333333334em, -23.4375em) {$\textcolor{black}{\bullet}$} ; 
\node[inner sep=5pt] (11) at (18.229166666666668em, -23.4375em) {$\textcolor{black}{\bullet}$} ; 
\node[inner sep=5pt] (12) at (23.4375em, -23.4375em) {$\textcolor{black}{\bullet}$} ; 
\node[inner sep=5pt] (13) at (7.8125em, -23.4375em) {$\bullet$} ; 
\node[inner sep=5pt] (14) at (46.90104166666667em, -39.31315104166667em) {$ $} ; 
\node[inner sep=5pt] (15) at (46.953125em, -16.708984375em) {$\eqref{eq:rew-mono-fo}$} ; 
\node[inner sep=5pt] (16) at (26.015625em, -2.203776041666667em) {$ $} ; 
\node[inner sep=5pt] (17) at (26.067708333333336em, -12.021484375em) {$\text{Apply-Hide } \eqref{eq:comp-assoc}$} ; 
\node[inner sep=5pt] (18) at (46.953125em, -28.141276041666668em) {$ $} ; 
\node[inner sep=5pt] (19) at (26.171875em, -18.427734375em) {$ $} ; 
\node[inner sep=5pt] (20) at (26.223958333333336em, -27.073567708333336em) {$\eqref{eq:rew-mono-fo}$} ; 
\node[inner sep=5pt] (21) at (33.85416666666667em, -23.4375em) {$\textcolor{black}{\bullet}$} ; 
\node[inner sep=5pt] (22) at (39.0625em, -23.4375em) {$\textcolor{black}{\bullet}$} ; 
\node[inner sep=5pt] (23) at (44.270833333333336em, -23.4375em) {$\textcolor{black}{\bullet}$} ; 
\node[inner sep=5pt] (24) at (28.645833333333336em, -23.4375em) {$\bullet$} ; 
\node[inner sep=5pt] (25) at (33.85416666666667em, -7.8125em) {$\textcolor{black}{\bullet}$} ; 
\node[inner sep=5pt] (26) at (39.0625em, -7.8125em) {$\textcolor{black}{\bullet}$} ; 
\node[inner sep=5pt] (27) at (44.270833333333336em, -7.8125em) {$\textcolor{black}{\bullet}$} ; 
\node[inner sep=5pt] (28) at (28.645833333333336em, -7.8125em) {$\bullet$} ; 
\path 
(1) to[fore, black,->, into, ] node[coordinate](29){} (2) 
(2) to[fore, black,->, into, ] node[coordinate](30){} (3) 
(4) to[fore, black,->, curve={ratio=0.2}, ] node[coordinate](31){} (1) 
(4) to[fore, black,->, curve={ratio=-0.30000000000000004}, ] node[coordinate](32){} (1) 
(4) to[fore, black,->, curve={ratio=-0.6}, ] node[coordinate](33){} (3) 
(1) to[fore, blue,->, curve={ratio=-0.5}, ] node[coordinate](34){} (3) 
(5) to[fore, black,-,] node[coordinate](35){} (6) 
(10) to[fore, blue,->, into, ] node[coordinate](36){} (11) 
(11) to[fore, black,->, into, ] node[coordinate](37){} (12) 
(13) to[fore, blue,->, curve={ratio=0.2}, ] node[coordinate](38){} (10) 
(13) to[fore, blue,->, curve={ratio=-0.30000000000000004}, ] node[coordinate](39){} (10) 
(13) to[fore, black,->, curve={ratio=-0.6}, ] node[coordinate](40){} (12) 
(13) to[fore, blue,->, curve={ratio=0.6}, ] node[coordinate](41){} (11) 
(16) to[fore, black,-,] node[coordinate](42){} (17) 
(19) to[fore, black,-,] node[coordinate](43){} (20) 
(21) to[fore, black,->, into, ] node[coordinate](44){} (22) 
(22) to[fore, black,->, into, ] node[coordinate](45){} (23) 
(24) to[fore, black,->, ] node[coordinate](46){} (21) 
(24) to[fore, black,->, curve={ratio=-0.6}, ] node[coordinate](47){} (23) 
(24) to[fore, black,->, curve={ratio=0.6}, ] node[coordinate](48){} (22) 
(25) to[fore, black,->, into, ] node[coordinate](49){} (26) 
(26) to[fore, blue,->, into, ] node[coordinate](50){} (27) 
(28) to[fore, black,->, curve={ratio=0.2}, ] node[coordinate](51){} (25) 
(28) to[fore, black,->, curve={ratio=-0.30000000000000004}, ] node[coordinate](52){} (25) 
(28) to[fore, orange,->, curve={ratio=-0.7}, ] node[coordinate](53){} (27) 
(28) to[fore, orange,->, curve={ratio=-0.5}, ] node[coordinate](54){} (26) 
(28) to[fore, orange,->, curve={ratio=0.5}, ] node[coordinate](55){} (26) 
(2) to[fore, blue,->, cell=0.05, ] node[coordinate](56){} (34) 
(1) to[fore, blue,->, cell=0.05, curve={ratio=-0.4}, ] node[coordinate](57){} (33) 
(1) to[fore, blue,->, cell=0.05, curve={ratio=0.1}, ] node[coordinate](58){} (33) 
(10) to[fore, blue,->, cell=0.05, curve={ratio=-0.2}, ] node[coordinate](59){} (41) 
(11) to[fore, black,->, cell=0.05, curve={ratio=-0.2}, ] node[coordinate](60){} (40) 
(10) to[fore, blue,->, cell=0.05, curve={ratio=0.4}, ] node[coordinate](61){} (41) 
(11) to[fore, black,->, cell=0.05, curve={ratio=0.2}, ] node[coordinate](62){} (40) 
(21) to[fore, black,->, cell=0.05, curve={ratio=-0.2}, ] node[coordinate](63){} (48) 
(22) to[fore, black,->, cell=0.05, curve={ratio=-0.2}, ] node[coordinate](64){} (47) 
(21) to[fore, black,->, cell=0.05, curve={ratio=0.4}, ] node[coordinate](65){} (48) 
(22) to[fore, black,->, cell=0.05, curve={ratio=0.2}, ] node[coordinate](66){} (47) 
(25) to[fore, red,->, cell=0.05, ] node[coordinate](67){} (54) 
(25) to[fore, red,->, cell=0.05, ] node[coordinate](68){} (55) 
(26) to[fore, orange,->, cell=0.05, curve={ratio=0.2}, ] node[coordinate](69){} (53) 
(26) to[fore, orange,->, cell=0.05, curve={ratio=0.5}, ] node[coordinate](70){} (53) 
; 
\path[->, transform shape, every edge quotes/.style={}, scale=0.7] 
(1) edge["${ }$" auto=left, pos=0.5, fore, black,->, into, ] (2) 
(2) edge["${ }$" auto=left, pos=0.5, fore, black,->, into, ] (3) 
(4) edge["${ }$" auto=left, pos=0.5, fore, black,->, curve={ratio=0.2}, ] (1) 
(4) edge["${ }$" auto=left, pos=0.5, fore, black,->, curve={ratio=-0.30000000000000004}, ] (1) 
(4) edge["${ }$" auto=left, pos=0.5, fore, black,->, curve={ratio=-0.6}, ] (3) 
(1) edge["${ }$" auto=left, pos=0.5, fore, blue,->, curve={ratio=-0.5}, ] (3) 
(5) edge["${ }$" auto=left, pos=0.5, fore, black,-,] (6) 
(10) edge["${ }$" auto=left, pos=0.5, fore, blue,->, into, ] (11) 
(11) edge["${ }$" auto=left, pos=0.5, fore, black,->, into, ] (12) 
(13) edge["${ }$" auto=left, pos=0.5, fore, blue,->, curve={ratio=0.2}, ] (10) 
(13) edge["${ }$" auto=left, pos=0.5, fore, blue,->, curve={ratio=-0.30000000000000004}, ] (10) 
(13) edge["${ }$" auto=left, pos=0.5, fore, black,->, curve={ratio=-0.6}, ] (12) 
(13) edge["${ }$" auto=right, pos=0.5, fore, blue,->, curve={ratio=0.6}, ] (11) 
(16) edge["${ }$" auto=left, pos=0.5, fore, black,-,] (17) 
(19) edge["${ }$" auto=left, pos=0.5, fore, black,-,] (20) 
(21) edge["${ }$" auto=left, pos=0.5, fore, black,->, into, ] (22) 
(22) edge["${ }$" auto=left, pos=0.5, fore, black,->, into, ] (23) 
(24) edge["${ }$" auto=left, pos=0.5, fore, black,->, ] (21) 
(24) edge["${ }$" auto=left, pos=0.5, fore, black,->, curve={ratio=-0.6}, ] (23) 
(24) edge["${ }$" auto=right, pos=0.5, fore, black,->, curve={ratio=0.6}, ] (22) 
(25) edge["${ }$" auto=left, pos=0.5, fore, black,->, into, ] (26) 
(26) edge["${ }$" auto=left, pos=0.5, fore, blue,->, into, ] (27) 
(28) edge["${ }$" auto=left, pos=0.5, fore, black,->, curve={ratio=0.2}, ] (25) 
(28) edge["${ }$" auto=left, pos=0.5, fore, black,->, curve={ratio=-0.30000000000000004}, ] (25) 
(28) edge["${ }$" auto=left, pos=0.5, fore, orange,->, curve={ratio=-0.7}, ] (27) 
(28) edge["${ }$" auto=right, pos=0.5, fore, orange,->, curve={ratio=-0.5}, ] (26) 
(28) edge["${ }$" auto=left, pos=0.5, fore, orange,->, curve={ratio=0.5}, ] (26) 
(2) edge["${ }$" auto=left, pos=0.5, fore, blue,->, cell=0.05, ] (34) 
(1) edge["${ }$" auto=left, pos=0.5, fore, blue,->, cell=0.05, curve={ratio=-0.4}, ] (33) 
(1) edge["${ }$" auto=left, pos=0.5, fore, blue,->, cell=0.05, curve={ratio=0.1}, ] (33) 
(10) edge["${ }$" auto=left, pos=0.5, fore, blue,->, cell=0.05, curve={ratio=-0.2}, ] (41) 
(11) edge["${ }$" auto=left, pos=0.5, fore, black,->, cell=0.05, curve={ratio=-0.2}, ] (40) 
(10) edge["${ }$" auto=left, pos=0.5, fore, blue,->, cell=0.05, curve={ratio=0.4}, ] (41) 
(11) edge["${ }$" auto=left, pos=0.5, fore, black,->, cell=0.05, curve={ratio=0.2}, ] (40) 
(21) edge["${ }$" auto=left, pos=0.5, fore, black,->, cell=0.05, curve={ratio=-0.2}, ] (48) 
(22) edge["${ }$" auto=left, pos=0.5, fore, black,->, cell=0.05, curve={ratio=-0.2}, ] (47) 
(21) edge["${ }$" auto=left, pos=0.5, fore, black,->, cell=0.05, curve={ratio=0.4}, ] (48) 
(22) edge["${ }$" auto=left, pos=0.5, fore, black,->, cell=0.05, curve={ratio=0.2}, ] (47) 
(25) edge["${ }$" auto=left, pos=0.5, fore, red,->, cell=0.05, ] (54) 
(25) edge["${ }$" auto=left, pos=0.5, fore, red,->, cell=0.05, ] (55) 
(26) edge["${ }$" auto=left, pos=0.5, fore, orange,->, cell=0.05, curve={ratio=0.2}, ] (53) 
(26) edge["${ }$" auto=left, pos=0.5, fore, orange,->, cell=0.05, curve={ratio=0.5}, ] (53) 
; 
\end{tikzpicture}
% END OF GENERATED LATEX
\]

\end{document}
\section{leftovers}
For example, if
Let us start with 
 \[
% YADE DIAGRAM two-mono.yade
% GENERATED LATEX
% \begin{tikzpicture}[every node/.style={outer sep=0pt,anchor=base,text height=1.2ex, text depth=0.25ex}] 
\node[inner sep=5pt] (0) at (11.904761904761905em, -7.142857142857143em) {$a$} ; 
\node[inner sep=5pt] (1) at (16.666666666666668em, -7.142857142857143em) {$b$} ; 
\node[inner sep=5pt] (2) at (21.428571428571427em, -7.142857142857143em) {$c$} ; 
\path 
(0) to[fore, black,->, into, ] node[coordinate](3){} (1) 
(1) to[fore, black,->, into, ] node[coordinate](4){} (2) 
; 
\path[->] 
(0) edge["${\scriptstyle f}$", pos=0.5, fore, black,->, into, ] (1) 
(1) edge["${\scriptstyle g}$", pos=0.5, fore, black,->, into, ] (2) 
; 
\end{tikzpicture}
% END OF GENERATED LATEX
\]
Prop

, creation of an initial workspace, and a reasoning mode.

In the first stage, the user defines the algebraic structure in which they want to reason about (for examples, the structure of categories).
In the second stage, 

The user can create a workspace, and add elements to it (e.g, drawing two monomorphisms).
\[
% YADE DIAGRAM two-mono.yade
% GENERATED LATEX
% \begin{tikzpicture}[every node/.style={outer sep=0pt,anchor=base,text height=1.2ex, text depth=0.25ex}] 
\node[inner sep=5pt] (0) at (11.904761904761905em, -7.142857142857143em) {$a$} ; 
\node[inner sep=5pt] (1) at (16.666666666666668em, -7.142857142857143em) {$b$} ; 
\node[inner sep=5pt] (2) at (21.428571428571427em, -7.142857142857143em) {$c$} ; 
\path 
(0) to[fore, black,->, into, ] node[coordinate](3){} (1) 
(1) to[fore, black,->, into, ] node[coordinate](4){} (2) 
; 
\path[->] 
(0) edge["${\scriptstyle f}$", pos=0.5, fore, black,->, into, ] (1) 
(1) edge["${\scriptstyle g}$", pos=0.5, fore, black,->, into, ] (2) 
; 
\end{tikzpicture}
% END OF GENERATED LATEX
\]
Then he enters a reasoning mode, where he can apply rewriting rules to modify the workspace, 
resulting in a new state (e.g., the composite arrow is a mono).
\[
% YADE DIAGRAM two-mono-composite-mono.yade
% GENERATED LATEX
\begin{tikzpicture}[every node/.style={outer sep=0pt,anchor=base,text height=1.2ex, text depth=0.25ex}] 
\node[inner sep=5pt] (0) at (11.904761904761905em, -7.142857142857143em) {$a$} ; 
\node[inner sep=5pt] (1) at (16.666666666666668em, -7.142857142857143em) {$b$} ; 
\node[inner sep=5pt] (2) at (21.428571428571427em, -7.142857142857143em) {$c$} ; 
\node[inner sep=5pt] (3) at (16.61904761904762em, -8.252232142857142em) {$=$} ; 
\path 
(0) to[fore, black,->, into, ] node[coordinate](4){} (1) 
(1) to[fore, black,->, into, ] node[coordinate](5){} (2) 
(0) to[fore, black,->, curve={ratio=0.4}, into, ] node[coordinate](6){} (2) 
; 
\path[->] 
(0) edge["${\scriptstyle f}$", pos=0.5, fore, black,->, into, ] (1) 
(1) edge["${\scriptstyle g}$", pos=0.5, fore, black,->, into, ] (2) 
(0) edge["${\scriptstyle g\circ f}$"', pos=0.5, fore, black,->, curve={ratio=0.4}, into, ] (2) 
; 
\end{tikzpicture}
% END OF GENERATED LATEX
\]

The different kinds of elements as well as the set of rewriting rules are defined a priori by the user.
The workspace is parameterised by two things:
\begin{itemize}
    
\item The kind of elements that can be added to the workspace is defined by a \emph{sort specification}.

Formally, this can be seen as type-theoretic list of dependent sorts, or as a finite direct category.
\item Then, the available rewriting rules are defined by a \emph{rewriting system specification}.
Both of them can be specified by the user. 
\end{itemize}

\end{document}


























































